%% \begin{deluxetable}{} command tell LaTeX how many columns
%% there are and how to align them.
\startlongtable
\begin{deluxetable}{cccc}
    
%% Keep a portrait orientation

%% Over-ride the default font size
%% Use Default (12pt)
\tabletypesize{\footnotesize}

%% Use \tablewidth{?pt} to over-ride the default table width.
%% If you are unhappy with the default look at the end of the
%% *.log file to see what the default was set at before adjusting
%% this value.

%% This is the title of the table.
\tablecaption{WASP-4b occultation times, uncertainties, and references.}
\label{tab:occultation_times}

%% This command over-rides LaTeX's natural table count
%% and replaces it with this number.  LaTeX will increment 
%% all other tables after this table based on this number
\tablenum{3}

%% The \tablehead gives provides the column headers.  It
%% is currently set up so that the column labels are on the
%% top line and the units surrounded by ()s are in the 
%% bottom line.  You may add more header information by writing
%% another line between these lines. For each column that requries
%% extra information be sure to include a \colhead{text} command
%% and remember to end any extra lines with \\ and include the 
%% correct number of &s.
\tablehead{
  \colhead{$t_{\rm occ}$ [BJD$_\mathrm{TDB}$]} &
  \colhead{$\sigma_{t_{\rm occ}}$ [days]} &
  \colhead{Epoch} & 
  \colhead{Reference}
}

%% All data must appear between the \startdata and \enddata commands
% XXX pasted in from selected_transit_times.tex
\startdata
 2455102.61210 &      0.00074 &    -511 &  \citet{caceres_ground-based_2011}\tablenotemark{a} \\
 2455172.20159 &      0.00130 &    -459 &      \citet{beerer_secondary_2011} \\
 2455174.87780 &      0.00087 &    -457 &      \citet{beerer_secondary_2011} \\
 2456907.88714 &      0.00290 &     838 &        \citet{zhou_secondary_2015}\tablenotemark{b} \\
\enddata

%% Include any \tablenotetext{key}{text}, \tablerefs{ref list},
%% or \tablecomments{text} between the \enddata and 
%% \end{deluxetable} commands

%% General table comment marker
\tablecomments{
	$t_{\rm occ}$ is the measured occultation midtime, minus the
	$2a/c=22.8$ second light travel time;
	$\sigma_{t_{\rm occ}}$ is the $1\sigma$ uncertainty on the occultation
	time.
}
\tablenotetext{a}{
\citet{caceres_ground-based_2011} reported this time in ``HJD'', with
an unspecified time standard. We assumed the time was originally in
${\rm HJD}_{\rm UTC}$, and converted to ${\rm BJD}_{\rm TDB}$ for the
tabulated time.
}
\tablenotetext{b}{
\citet{zhou_secondary_2015} fixed the epoch, and let $e\cos\omega$
float. Using the reported dates of observation, we converted their
$e\cos\omega$ values into an occultation time using
Equation~\ref{eq:occultation_time} of the text. 
}

\end{deluxetable}
