%\renewcommand{\arraystretch}{1.0}

\startlongtable
\begin{deluxetable*}{ccc}

%\tabletypesize{\small}

\tablenum{4}

%\tablewidth{0pt}

\tablecaption{Adopted stellar and planetary parameters.}
\label{tab:stellar_parameters}

\tablehead{
\colhead{Parameter} & \colhead{Value} & \colhead{Uncertainty} 
}

\startdata
Period $P$(days) & 1.33823251 & 0.00000031 \tablenotemark{a}  \\
Minimum reference Time $T_{0}$($BJD_{TDB}$) & 2454697.797973 & 0.000076    \\
Inclination $i$($\degr$) & 86.85  & 1.76   \\
Stellar Radius $R_{\star}$($R_{\sun}$) & 0.92 & 0.06   \\ 
Stellar Mass $M_{\star}$($M_{\sun}$) & 0.89  &  0.01   \\
Stellar gravity $\log g_{\star}$($cm/s$) & 4.461 &  0.054 \\
Semimajor axis $a$($UA$) & 0.0228  & 0.00013   \\
Age (Gyr) & 7.0 & 2.9   \\  
Stellar effective temperature $T_{eff}$(K) & 5436 & 34  \\
Metallicity $[Fe/H]$(dex) & -0.05 & 0.04   \\
Planet Radius $R_{P}$($R_{Jup}$) & 1.33  &  0.16  \\
Planet Mass $M_{P}$($M_{Jup}$) & 1.216 &   0.013  \\
Planet surface gravity $g_{P}$($m/s^{2}$) & 16.41 &  2.49 \\
Planet equilibrium temperature $T'_{eq}$(K) & 1664 & 54   \\
\enddata
\tablecomments{
    $t_{\rm tra}$ is the measured transit midtime, and $\sigma_{t_{\rm tra}}$ is its
    $1\sigma$ uncertainty.
    $\sigma_{t_0}$ was evaluated from the sampled posteriors by taking
    the maximum of the difference between the 84th percentile
    minus the median, and the median minus the 16th percentile.
    The ``Reference'' column refers to the work describing the
    original observations.
    The ``H13?'' column is 1 if the mid-time value was taken from 
    \citet{hoyer_tramos_2013}.  Otherwise, the mid-time
    came from the column listed in ``Reference''.
}
\tablenotetext{a}{
  %FIXME cite etc.
The numbers in parenthesis give the 1$\sigma$ uncertainty in the final
two digits, where appropriate.
}

\end{deluxetable*}
