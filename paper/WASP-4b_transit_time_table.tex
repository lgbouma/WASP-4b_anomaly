%% \begin{deluxetable}{} command tell LaTeX how many columns
%% there are and how to align them.
\startlongtable
\begin{deluxetable*}{ccccc}
    
%% Keep a portrait orientation

%% Over-ride the default font size
%% Use Default (12pt)
\tabletypesize{\scriptsize}

%% Use \tablewidth{?pt} to over-ride the default table width.
%% If you are unhappy with the default look at the end of the
%% *.log file to see what the default was set at before adjusting
%% this value.

%% This is the title of the table.
\tablecaption{WASP-4b transit times.}
\label{tab:transit_times}

%% This command over-rides LaTeX's natural table count
%% and replaces it with this number.  LaTeX will increment 
%% all other tables after this table based on this number
\tablenum{1}

%% The \tablehead gives provides the column headers.  It
%% is currently set up so that the column labels are on the
%% top line and the units surrounded by ()s are in the 
%% bottom line.  You may add more header information by writing
%% another line between these lines. For each column that requries
%% extra information be sure to include a \colhead{text} command
%% and remember to end any extra lines with \\ and include the 
%% correct number of &s.
\tablehead{
  \colhead{$t_{\rm tra}$ [BJD$_\mathrm{TDB}$]} &
  \colhead{$\sigma_{t_{\rm tra}}$ [days]} &
  \colhead{Epoch} & 
  \colhead{Time Reference} & 
  \colhead{Observation Reference}
}

%% All data must appear between the \startdata and \enddata commands
% XXX pasted in from selected_transit_times.tex
\startdata
 2454368.59279 &      0.00033 &   -1354 &       \citet{hoyer_tramos_2013} &           \citet{wilson_wasp-4b_2008} \\
\enddata

%% Include any \tablenotetext{key}{text}, \tablerefs{ref list},
%% or \tablecomments{text} between the \enddata and 
%% \end{deluxetable} commands

%% General table comment marker
\tablecomments{
Table~1 is published in its entirety in a machine-readable format.
The first row is shown for guidance regarding form and
content.  $t_{\rm tra}$ is the measured transit midtime, and
$\sigma_{t_{\rm tra}}$ is its $1\sigma$ uncertainty.  ``Time
Reference'' refers to the provenance of the timing measurement, which
may differ from the ``Observation Reference'' in cases for which a
homogeneous timing analysis was performed.  The
\citealt{hoyer_tramos_2013} BJD$_{\rm TT}$ times are equal to
BJD$_{\rm TDB}$ for our purposes \citep{urban_explanatory_2012}.  We
omitted the timing measurements from
\citet{southworth_high-precision_2009}, since there were technical
problems with the computer clock at the time of
observation~\citep{nikolov_wasp-4b_2012}.  The two Baxter et al.\ (in
prep) times were obtained from Spitzer/IRAC transit light curves in
the 3.6$\mu$m and 4.5$\mu$m channels.
}
\vspace{-1.5cm}
\end{deluxetable*}
