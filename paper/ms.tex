% TODO:
%
% * email to full collaborator list
% * add any missing CPS coauthors
%
%%%%%%%%%%%%%%%%%%%%%%%%%%%%%%%%%%%%%%%%%%%%%%%%%%%%%%%%%%%%%%%%%%%%%%%%%%%%%%%

\documentclass[RNAAS]{aastex62}
\usepackage{amsmath,amstext,amssymb}
\usepackage[T1]{fontenc}
\usepackage{apjfonts}
\usepackage[figure,figure*]{hypcap}
\usepackage{graphics,graphicx}
\usepackage{hyperref}

\begin{document}

% NOTE: may want to improve title
\title{WASP-4 is accelerating towards the Earth}

\correspondingauthor{L. G. Bouma}
\email{luke@astro.princeton.edu}

%
% key authors: presumably Rachel did the Zorro reductions?
%
\author[0000-0002-0514-5538]{L. G. Bouma}
\affiliation{ Department of Astrophysical Sciences, Princeton
University, 4 Ivy Lane, Princeton, NJ 08540, USA}
%
\author[0000-0002-4265-047X]{J. N. Winn}
\affiliation{ Department of Astrophysical Sciences, Princeton
University, 4 Ivy Lane, Princeton, NJ 08540, USA}
%
\author{R. A. Matson}
\affiliation{NASA Ames Research Center, Moffett Field, CA 94035, USA}
%

%
% contributing authors: alphabetical
%
\author[0000-0002-0531-1073]{H. Isaacson}
\affiliation{Astronomy Department, University of California, Berkeley,
CA 94720, USA}
%
\author[0000-0001-8638-0320]{A. W. Howard}
\affiliation{Cahill Center for Astrophysics, California Institute of
Technology, Pasadena, CA 91125, USA}
%
\author{S. B. Howell}
\affiliation{NASA Ames Research Center, Moffett Field, CA 94035, USA}
%
\author{H. Knutson}
\affiliation{Division of Geological and Planetary Sciences, California
Institute of Technology, Pasadena, CA 91125, USA}
%

%% Note that RNAAS manuscripts DO NOT have abstracts.
%% See the online documentation for the full list of available subject
%% keywords and the rules for their use.
\keywords{Exoplanet tides (497), Exoplanet dynamics (490), Radial velocity 
(1332), Transit timing variation method (1710)}

%% Start the main body of the article. If no sections in the 
%% research note leave the \section call blank to make the title.

%%%%%%%%%%%%%%%%%%%%%%%%%%%%%%%%%%%%%%%%%%%%%%%%%%%%%%%%%%%%%%%%%%%%%%%%%%%%%%%

\section{Introduction}

Since the first discovery of transiting hot Jupiters, the hunt for
tidal orbital decay has received much attention (CITE LIKE 20 PAPERS).
These efforts are gradually beginning to yield fruit: by combining
transit and occultation timing and radial velocity measurements, CITET
Yee have shown that the data for the hot Jupiter WASP-12b are most
compatible with orbital decay.

This study highlights a point that, while obvious, has perhaps not
yet received due attention.
The point is that the hunt for orbital decay of hot Jupiters will be
crippled without long-term radial velocity monitoring programs.
The reason is simple: when the timescale of orbital period change far
exceeds the observing baseline, the first
deviation from constant periodicity is always quadratic in the transit
number:
\begin{align}
  t_{\rm tra} = t_0 + P E + \frac{1}{2}\frac{dP}{dE}E^2,
\end{align}
where
\begin{align}
  \frac{dP}{dE} = P \frac{dP}{dt},
\end{align}
$E$ is the transit number, $P$ is the orbital period, $t_0$ is the
reference epoch, and $t_{\rm tra}$ is the transt mid-time.

There are ample reasons to expect orbital decay to be common
(CITE).
However there are reasons to expect the R\'omer delay to perhaps be
even {\it more} common (CITE Knutson 14, Bryan 16).
Both manifest to first order identically in transit times (up to the
sign of the line-of-sight acceleration) -- as a non-zero period
derivative.

Specifically, \citet{knutson_friends_2014} showed that X\% of hot
Jupiters show some form of radial acceleration in radial
velocity time-series with baselines of YY years.
Assuming a separation distribution of (WHATEVER), this implies that
ZZ\% of hot Jupiters will {\it always} show a changing orbital period,
over timescale of say 10 years.

This is specific to hot Jupiters. The abundance of giant planets
exterior to Super-Earth sized systems is also high (CITE Bryan 2019).

The main example of this study is the hot Jupiter WASP-4b which has an
orbital period that from transits appears to be decreasing by about 10
milliseconds per year.
After discovering the timing variations in the TESS data, we obtained
N additional radial velocity measurements using Keck-HIRES,
extending the RV baseline by Y years.
Previously, the radial velocities were sparse and were consistent with
not showing any linear trend.
Our new measurements reveal a line of sight acceleration of
$\dot{\gamma} = -XX {\rm m\,s^{-1}\,yr^{-1}}$.
This translates to an expected period decrease simply from the light
travel time effect of ZZ milliseconds per year --- about what is
observed.

\section{WASP-4's acceleration towards the Earth}

Caveats: we of course checked the RV vs bisector scatter diagram, as
well as the RV vs other activity indicators, to ensure we weren't
seeing a spurious multi-hundred meter per second change.

\section{Do other hot Jupiters show the Romer delay?}
yes, HAT-P-7b. but omg utc to tdb.


\section{What is the expected Pdot distribution?}






SUMMARY OF LITERATURE

This is a major challenge in the hunt for orbital decay.  The main
example of this study is the hot Jupiter WASP-4b, which we recently
showed transited $\approx$82 seconds early for the TESS spacecraft
assuming a constant orbital period \citep{bouma_wasp-4b_2019}.  Our
timing analysis included data from peer-reviewed literature for which
the times were measured from a single transit, and for which the
midpoint was allowed to be a free parameter. We also required the time
system to be clearly documented\footnote{In other words, we needed to
know whether any heliocentric or barycentric corrections had been
performed, and whether the absolute time system was UTC or TDB.}. The
transit times we analyzed spanned 2007 to 2019
\citep{wilson_wasp-4b_2008,gillon_improved_2009,winn_transit_2009,dragomir_terms_2011,sanchis-ojeda_starspots_2011,nikolov_wasp-4b_2012,hoyer_tramos_2013,ranjan_atmospheric_2014,huitson_gemini_2017}.
The combined timing data were best fit by an ephemeris with a constant
negative period derivative; our best-fit decay rate was $\dot{P} =
-12.6 \pm 1.2$ milliseconds per year.  Our interpretation was that the
apparent period change could be caused by any of three scenarios: a
decaying orbit, a precessing orbit, and an orbit being gravitationally
perturbed by an outer companion.

Thereafter, \citet{southworth_transit_2019} reported 22 new transit
times for the system, and confirmed that the updated series of transit
times was consistent with a quadratic ephemeris.  Their interpretation
of the timing variations did not differ in any major respects from our
own, though with additional data they found a lower best-fit decay
rate of $\dot{P} = -XX.X \pm X.X$ milliseconds per year.

A separate study by \citet{baluev_homogeneously_2019} reported
additional light curves. \citeauthor{baluev_homogeneously_2019}
analyzed their newly obtained photometry, along with archival light
curves that we omitted from our analysis due to systematic
uncertainties in the absolute time system.
\citeauthor{baluev_homogeneously_2019} found that when they used all
the available TTV data, the need for a quadratic ephemeris was present
``at the high $\sim 5-7$ sigma level''.  However, they pointed out
that if they used lower-precision subsets of the available timing
data, the necessity for the quadratic term decreased.
\citeauthor{baluev_homogeneously_2019} also pointed out that the
precise transit times reported by \citet{huitson_gemini_2017} were
quite important in the time-series.  Overall,
\citeauthor{baluev_homogeneously_2019} did not find the claim of a
period decrease convincing.

One line of follow-up needed to confirm and understand the timing
variation was additional radial velocity observations.  The longest RV
baseline previously available was six observations acquired by
\citet{knutson_friends_2014} and the CKS team on Keck HIRES from
2010 to 2013.  This past season we acquired four additional
observations with Keck HIRES. Even before fitting out the hot Jupiter,
the residuals showed a strong linear trend.

For completeness, in our analysis we included CORALIE measurements
from \citet{wilson_wasp-4b_2008} and \citet{triaud_spin-orbit_2010},
using the homogeneous velocities reported by the latter authors.  We
included the HARPS values reported by \citet{pont_determining_2011},
which are identical to those from \citet{husnoo_observational_2012}.
We omitted the HARPS data points taken over three nights by
\citet{triaud_spin-orbit_2010} for Rossiter-McLaughlin observations
because they were calculated using a different pipeline than the
longer-baseline \citeauthor{pont_determining_2011} measurements, and
necessarily inclusion of an extra offset term would nullify their
statistical value.

We then fitted a single Keplerian orbit, plus instrument offsets,
jitters, and a linear trend
\citep[][\texttt{radvel}]{fulton_radvel_2018}.  We set Gaussian priors
on the period and time of inferior conjunction using the values from
\citet{bouma_wasp-4b_2019}, and fixed the eccentricity to zero,
consistent with results from \citet{beerer_secondary_2011},
\citet{knutson_friends_2014} and \citet{bonomo_gaps_2017}.  The
remaining free parameters were the velocity semi-amplitude, the
instrument zero-points, the instrument jitters (an additive white
noise term for each instrument), and optionally a linear acceleration
term ($\dot{v_{\rm r}}$).

%FoHJ: gammadot = -0.0099^{+0.0052}_{-0.0054}
The AIC and BIC strongly favored a model with a linear trend ($\Delta
{\rm BIC}=73$ compared to a model without the linear trend).
Figure~\ref{fig:rv_o_minus_c} shows the best-fitting model, with the
orbit of WASP-4b subtracted.  The best-fit radial velocity derivative,
$\dot{v}_{\rm r}$, is
\begin{equation}
  \dot{v}_{\rm r} = -0.0422^{+0.0028}_{-0.0027}\,{\rm m\,s^{-1}\,day^{-1}}.
\end{equation}
This acceleration towards our line of sight is 4.3 times faster than
the $\approx$$2\sigma$ trend \citet{knutson_friends_2014}
found from the shorter baseline.  The radial velocity values are
available in the data-behind-the-figure online version of this {\it
Note}.

Under the assumption of constant acceleration, $\dot{P} = \dot{v}_{\rm
r} P / c$, the implied period decrease that should be seen in
transits is roughly $5.9 \pm 0.4$ milliseconds per year.  All the
available TTV data show a period decrease of $\approx 8-12$
milliseconds per year~\citep{bouma_wasp-4b_2019,southworth_transit_2019,baluev_homogeneously_2019}.

Though the quantitative agreement is not perfect, Occam's razor would
suggest that the line of sight acceleration is probably a sufficient
explanation for the apparent decrease of WASP-4b's orbital period.
While further radial velocity observations of the system should help
in determining the mass and semi-major axis of the companion, it seems
unlikely that additional transit observations will yield near-term
constraints on orbital decay.


%%%%%%%%%%%%%%%%%%%%%%%%%%%%%%%%%%%%%%%%%%%%%%%%%%%%%%%%%%%%%%%%%%%%%%%%%%%%%%%
% RNAAS: you get one figure.

\begin{figure}
    \begin{center}
		\includegraphics{rv_fit.pdf}
    \end{center}
    \vspace{-0.8cm}
    \caption{
      {\bf Radial velocity observations of WASP-4.}
      The orbit of WASP-4b has been subtracted.  The best-fit linear
      trend from the RVs is shown with the black line, with $1\sigma$
      errors in gray. The trend needed to produce the period decrease
      of 12.6 milliseconds per year seen by \citet{bouma_wasp-4b_2019}
      in transits is the purple dotted line in the bottom left.
     \label{fig:rv_o_minus_c}
    }
\end{figure}

%%%%%%%%%%%%%%%%%%%%%%%%%%%%%%%%%%%%%%%%%%%%%%%%%%%%%%%%%%%%%%%%%%%%%%%%%%%%%%%

\acknowledgements
%
This paper includes data collected by the TESS mission, which are
publicly available from the Mikulski Archive for Space Telescopes
(MAST).
%
Funding for the TESS mission is provided by NASA's Science Mission
directorate.
%
This work made use of NASA's Astrophysics Data System Bibliographic
Services.
%
This research has made use of the VizieR catalogue access tool, CDS,
Strasbourg, France. The original description of the VizieR service was
published in A\&AS 143, 23.
%
This work has made use of data from the European Space Agency (ESA)
mission {\it Gaia} (\url{https://www.cosmos.esa.int/gaia}), processed
by the {\it Gaia} Data Processing and Analysis Consortium (DPAC,
\url{https://www.cosmos.esa.int/web/gaia/dpac/consortium}). Funding
for the DPAC has been provided by national institutions, in particular
the institutions participating in the {\it Gaia} Multilateral
Agreement.
%
\newline
%
\software{
	\texttt{astroplan} \citep{astroplan2018},
  \texttt{radvel} \citep{fulton_radvel_2018}
  %FIXME numerical python packages
}

\bibliographystyle{yahapj}                            
\bibliography{bibliography} 

\end{document}
