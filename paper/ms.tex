%\documentclass[preprint2,tighten]{aastex6}
\documentclass[12pt,twocolumn,tighten]{aastex62}
%\pdfoutput=1 %for arXiv submission
\usepackage{amsmath,amstext,amssymb}
\usepackage[T1]{fontenc}
\usepackage{apjfonts}
\usepackage[figure,figure*]{hypcap}
\usepackage{graphics,graphicx}
\usepackage{hyperref}
\usepackage{comment}
\usepackage{todonotes}

\renewcommand*{\sectionautorefname}{Section} %for \autoref
\renewcommand*{\subsectionautorefname}{Section} %for \autoref

%% Reintroduced the \received and \accepted commands from AASTeX v5.2.
%% Add "Submitted to " argument.
\received{\today}
\revised{---}
\accepted{---}
\submitjournal{AAS journals.}

\shortauthors{Bouma et al.}
\shorttitle{Decreasing Period of WASP-4\lowercase{b}}

\begin{document}

\title{The Decreasing Orbital Period of WASP-4\lowercase{b}}

\correspondingauthor{L. G. Bouma}
\email{luke@astro.princeton.edu}

\author[0000-0002-0514-5538]{L. G. Bouma}
\affiliation{ Department of Astrophysical Sciences, Princeton
University, 4 Ivy Lane, Princeton, NJ 08540, USA}

\author[0000-0002-4265-047X]{J. N. Winn}
\affiliation{ Department of Astrophysical Sciences, Princeton
University, 4 Ivy Lane, Princeton, NJ 08540, USA}

\author[0000-0002-6939-9211]{T. Daylan}
\affiliation{ Department of Physics and Kavli Institute for Astrophysics
and Space Research, Massachusetts Institute of Technology, Cambridge, MA
02139, USA }

\author[0000-0002-3481-9052]{K. G. Stassun}
\affiliation{Vanderbilt University, Department of Physics \& Astronomy,
6301 Stevenson Center Lane, Nashville, TN 37235, USA}
\affiliation{Fisk University, Department of Physics, 1000 17th Avenue
N., Nashville, TN 37208, USA}

\author[0000-0002-7084-0529]{S. Kane}
\affiliation{Department of Earth Sciences, University of California,
Riverside, CA 92521, USA}
 
%-------------------------------------
% TESS Mission Architects:
% These authors should be listed in this order
% see https://spacebook.mit.edu/pages/viewpage.action?pageId=24543276
%-------------------------------------
%\\ \textcolor{red}{6 TESS Mission Architects:}
\author{G. R. Ricker} % grr@space.mit.edu WAITING
\affiliation{ Department of Physics and Kavli Institute for Astrophysics
and Space Research, Massachusetts Institute of Technology, Cambridge, MA
02139, USA }
\author[0000-0001-6763-6562]{R. Vanderspek} % roland@space.mit.edu WAITING
\affiliation{ Department of Physics and Kavli Institute for Astrophysics
and Space Research, Massachusetts Institute of Technology, Cambridge, MA
02139, USA }
\author[0000-0001-9911-7388]{D. W.~Latham} % dlatham@cfa.harvard.edu WAITING
\affiliation{ Harvard-Smithsonian Center for Astrophysics, 60 Garden
Street, Cambridge, MA 02138, USA }
\author{S. Seager} % seager@mit.edu WAITING
\affiliation{ Department of Earth, Atmospheric, and Planetary Sciences,
Massachusetts Institute of Technology, Cambridge, MA 02139, USA }
\author[0000-0002-4715-9460]{J. M.~Jenkins} % jon.jenkins@nasa.gov WAITING
\affiliation{ NASA Ames Research Center, Moffett Field, CA 94035, USA }
%-------------------------------------
% 3 representatives of each of SPOC, POC, TSO, for a total of 9. 
%These 9 authors should be listed in alphabetical order
%-------------------------------------
%\\ \textcolor{red}{3 TSO coauthors:}
\author{TSO1}
\author{TSO2}
\author{TSO3}
%\\ \textcolor{red}{3 SPOC coauthors:}
\author{J. D. Twicken} % Joseph.Twicken@nasa.gov WAITING
\affiliation{ NASA Ames Research Center, Moffett Field, CA 94035, USA }
\affiliation{ SETI Institute, Mountain View, CA, 94043}
\author{B. Wohler} % bill.wohler@nasa.gov WAITING
\affiliation{ NASA Ames Research Center, Moffett Field, CA 94035, USA }
\affiliation{ SETI Institute, Mountain View, CA, 94043}
\author{D. A. Caldwell} % douglas.Caldwell@nasa.gov % WAITING
\affiliation{ NASA Ames Research Center, Moffett Field, CA 94035, USA }
\affiliation{ SETI Institute, Mountain View, CA, 94043}
%\\ \textcolor{red}{3 POC coauthors:}
\author{POC1}
\author{POC2}
\author{POC3}


\begin{abstract}
  %TODO: final numbers
  Combining data from the Transiting Exoplanet Survey Satellite (TESS)
  with previous studies, we show that the transit times of WASP-4b are
  incompatible with a constant orbital period. 
  In particular, the transits seem to have arrived $77.8 \pm 10.7$
  seconds early, and the period appears to be shrinking by
  $\dot{P}=-12.1 \pm 1.2$ milliseconds per year.   
  From TESS observations of WASP-6b, WASP-18b, and WASP-46b, we show
  that a systematic offset between the TESS time stamps and the barycentric
  reference sufficient to explain WASP-4b is ruled out at 6.3$\sigma$.
  If the timing variations are astrophysical,
  major contributors to the period change could include both
  apsidal precession and tidal decay.
  The Doppler shift from WASP-4's acceleration towards us could
  account for at most one quarter of the observed period change
  (at $2\sigma$).
  Further transit and occultation studies will help confirm the
  reality of the timing variation, and eventually determine its cause.
\end{abstract}

\keywords{
  planet-star interactions ---
  planets and satellites: individual (WASP-4b, WASP-5b, WASP-6b,
    WASP-12b, WASP-18b, WASP-46b) ---
  binaries: close
}

\section{Introduction}
\label{sec:intro}

Long-term monitoring of hot Jupiter transit and occultation times should
eventually reveal two distinct processes.  First, whenever the summed
spin and orbital angular momenta in a hot Jupiter system are below a
particular critical value, the system will be unstable to tidal decay
\citep{counselman_outcomes_1973,hut_stability_1980}.  Most hot Jupiters
satisfy this condition; their orbits should be shrinking
\citep{levrard_falling_2009,matsumura_tidal_2010}.  The rate of decay is
uncertain, and depends on how friction dissipates energy carried by
gravitational tides (\citealt{ogilvie_tidal_2014} gives a review).  If
we could directly measure orbital decay rates, it would inform our
understanding of tidal dissipation.  This is important because the
dissipation rates determine the fate of the system.  During its
inspiral, if the planet fills its Roche lobe below the stellar surface,
the drag force from the stellar envelope accelerates the inspiral,
leading to a `direct-impact' merger
\citep{metzger_optical_2012,macleod_planetary_2018} Alternatively, if
the planet fills its Roche lobe above above the stellar surface, then it
could either be dynamically disrupted into an accretion disk or it could
slowly transfer mass to the star
\citep{metzger_optical_2012,valsecchi_tidally-driven_2015,jackson_tidal_2016}.
Further afield, the outcome of the eventual interaction between the
Earth and red giant Sun also hinges on the efficiency of tidal
dissipation \citep{rasio_tidal_1996}.

The second process that long-term timing studies should reveal is
rotation of the orbital ellipse within the orbital plane (apsidal
precession).  This effect has been studied extensively in eclipsing
binaries \citep[{\it e.g.},][]{russell_notes_1939,
schwarzschild_structure_1958,borkovits_eclipse_2015}, and has been
explored theoretically for transiting planets
\citep{heyl_using_2007,pal_periastron_2008,jordan_observability_2008,ragozzine_probing_2009}.
However, apsidal precession has yet to be directly observed in hot
Jupiters, and the orbits would need non-zero eccentricity for it to be
detectable.  Though the eccentricities of hot Jupiters are expected to
be small, there are many mechanisms that could pump them away from
circularity (see \S~\ref{sec:apsidal_precession}, and
\citealt{bailey_understanding_2019}).  For hot Jupiters, the apsidal
precession is usually dominated by the quadrupolar distortion of the
planet by the star's tidal force \citep{ragozzine_probing_2009}.  If we
measured apsidal precession in a hot Jupiter system, we could thus use
the observed precession rate to compute the planet's Love number, giving
a third parameter along with the mass and radius with which to describe
the planet's interior structure \citep[{\it e.g.},][who performed a
similar procedure for HAT-P-13b]{batygin_determination_2009}.  Measuring
small eccentricities and understanding their origin would also help us
understand the dynamical histories and interior structures of hot
Jupiters.  \citep[{\it
e.g.},][respectively]{dawson_origins_2018,ibgui_tidal_2010}

While numerous indirect studies have called attention to
population-level effects of tidal decay
\citep{jackson_observational_2009,hansen_calibration_2010,penev_constraining_2012,husnoo_observational_2012,matsakos_origin_2016,cameron_hierarchical_2018,penev_empirical_2018},
the most convincing direct evidence to date for either orbital decay or
apsidal precession has come from WASP-12b.
\citet{maciejewski_departure_2016} showed that the transit times of
WASP-12b did not follow a linear ephemeris at $5\sigma$ confidence, and
measured $\dot{P} = -26 \pm 4\,{\rm ms}\,{\rm yr}^{-1}$.
\citet{patra_2017} presented new transit and occultation times for the
system, and found $\dot{P} = -29 \pm 3\,{\rm ms}\,{\rm yr}^{-1}$.
Assuming that the period change was caused entirely by tidal decay,
\citet{patra_2017} found values for the stellar quality factor of
$Q_\star' \approx 2\times10^5$.  This level of dissipation cannot be
explained by standard equilibrium tide
models~\citep{penev_tidal_2011,ogilvie_tidal_2014}.
\citet{weinberg_tidal_2017} found though that if WASP-12 were a
subgiant, tidal dissipation due to a dynamical tide might be sufficient
to produce the required level of friction.
\cite{bailey_understanding_2019} concurred, but noted that the
observational constraints on WASP-12 itself favor a main-sequence star,
not a subgiant.  More data are needed to determine the cause of the
timing variations in WASP-12b.

This work focuses on WASP-4b, which is also a plausible candidate for
orbital decay.  The planet has radius $1.37 \pm 0.04\ R_{\rm Jup}$, and
mass $1.29 \pm 0.10\ M_{\rm Jup}$ \citep{southworth_homogeneous_2011}.
It has an orbital period of 1.34 days, orbits its host star at a
distance of $a/R_\star = 5.41$, and has been frequently observed over
the past decade \citep{wilson_wasp-4b_2008,huitson_gemini_2017}.  The
orbit is consistent with circular \citep[][but see
\S~\ref{sec:apsidal_precession}]{beerer_secondary_2011,husnoo_observational_2012,bonomo_gaps_2017}.
The host star is a G7V dwarf, with $R_\star = 0.92 \pm 0.13\ R_\odot$,
$M_\star = 0.92 \pm 0.07\ M_\odot$, $T_{\rm eff} = 5400 \pm 90\ {\rm
K}$, and an isochronal age of roughly $7\ {\rm Gyr}$
\citep{southworth_homogeneous_2011,petrucci_no_2013,doyle_accurate_2013}.

We measure new transit times for WASP-4b using data from TESS
(\S~\ref{sec:transits}).  We then merge these times with previous
observations and compare three models for the times
(\S~\ref{sec:timing}): a constant period, a decaying period, and a
slightly eccentric orbit undergoing precession.  We rule out a constant
period, but cannot firmly distinguish between the latter two
possibilities.  Each would have interesting implications
(\S~\ref{sec:implications}).  We rule out the possibility that a
systematic time system offset causes the observed variation
(Appendix~\ref{sec:verify_tess}), and conclude by advocating for further
monitoring of the system (\S~\ref{sec:future}).



\section{New transit times}
\label{sec:transits}

\begin{figure}[t]
    \begin{center}
        \includegraphics[width=0.48\textwidth]{f1.png}
    \end{center}
    \vspace{-0.5cm}
    \caption{
        {\bf TESS transits of WASP-4b.} On the left, black points are
        TESS flux measurements, with a vertical offset applied. Blue
        curves are best-fit models. Rounded midtimes are printed in
        BTJD next to the appropriate transits.  The residuals are
        shown on the right.
        \label{fig:lightcurves}
    }
\end{figure}

\subsection{Observations and cleaning}
WASP-4 was observed in TESS camera 2 from August 23, 2018 to September
20, during the spacecraft's second sector of science operations.  The
star is designated by the TESS Input Catalog as TIC 402026209
\citep{stassun_TIC_2018}.

The TESS spacecraft recorded images of WASP-4 in an $11\times11$ pixel
array every 2 minutes.  It was included on the ``short-cadence'' target
list thanks to the Guest Investigator programs of J.\ Southworth and S.\
Kane (G011112 and G011183 respectively).  After being downlinked through
the Deep Space Network, the data were processed through the Science
Processing Operations Center (SPOC) pipeline at NASA
Ames~\citep{jenkins_tess_2016}.  One important step in this processing
was converting from spacecraft time to barycentric time.  SPOC then
reports timestamps in ``BTJD'', where ${\rm BTJD} = {\rm BJD} -
2457000$, BJD is the barycentric julian date, and all times are in the
TDB reference (see \citealt{urban_explanatory_2012} for explanations of
time systems).  The lightcurves were then vetted and released by the MIT
TESS Science Office to the Mikulski Archive for Space
Telescopes~\citep{ricker_tess_alerts_2018}.

We begin our analysis with the Presearch Data Conditioning (PDC) light
curves from MAST.  The steps used to find an optimal aperture (for
WASP-4, a $3\times 3$ pixel array) and the PDC-MAP algorithm are
described by \citet{smith_kepler_apertures_2017} and
\citet{smith_kepler_PDC_2017} respectively.

We then processed the lightcurve as follows.  First, we removed all points
with non-zero quality flags.  This removes data contaminated by
coarse spacecraft pointing, cosmic rays, and other annoyances
\citep{tess_data_product_description_2018}.  We then filtered out known
bad observing windows.  The data closest to spacecraft perigee typically
show ramp-like systematics, so we clipped out the first and last hour of
both orbits.  Next, we focused on the momentum dumps. As described in
the data release
notes\footnote{\url{archive.stsci.edu/hlsps/tess-data-alerts/hlsp_tess-data-alerts_tess_phot_s01_tess_v1_release-notes.pdf}},
during the first two TESS sectors, every 2.5~days the spacecraft's
reaction wheels were reset to maintain pointing stability.  These events
were assigned quality flags corresponding to ``Reaction Wheel
Desaturation Event'' and ``Manual Exclude''.  For WASP-4,
these flags were simultaneously set for 54 distinct cadences, and there
were 10 momentum dumps, averaging about 10 minutes of flagged data per
dump.  Simply out of caution, we clipped out an additional 10 minutes
before and after every momentum dump.  Before applying any filters, we
had 19737 data points. After applying all the filters, we were left with
18165 measurements, or 92\% of the original data.

After ``cleaning'' the data, we normalized the flux measurements by
dividing out the median flux.  We then converted the timestamps from
BTJD to BJD by adding the appropriate 2457000 day offset
\citep{tess_data_product_description_2018}.  Many of these and
subsequent processing steps were performed using
\texttt{astrobase}~\citep{bhatti_astrobase_2018} We opted not to
``flatten'' the lightcurves at this stage, as is often done with {\it
e.g.}, a spline or gaussian process regression.  Such procedures are
asymmetric about the transit minimum and can skew the transit time
measurement without being captured by the measurement uncertainties.



\subsection{Measuring the transit times}


\begin{figure}[t]
    \begin{center}
        \leavevmode
        \includegraphics[width=0.48\textwidth]{f2.pdf}
    \end{center}
    \vspace{-0.5cm}
    \caption{
      %TODO: final number
        {\bf TESS saw WASP-4b transit earlier than expected.}
        Points are differences between observed transit times and the
        constant-period ephemeris we fit to literature transits.
        Black points are the more precise half of literatures times;
        gray points are the less precise half.  The blue curves
        represent the $\pm 1\sigma$ credible interval around the
        constant-period ephemeris.  The binned TESS observation (red
        point) arrived $77.8 \pm 10.7\ {\rm seconds}$ before it was
        expected.
        \label{fig:arrived_early}
    }
\end{figure}

Using the times, flux measurements, and errors determined above, we ran
Box Least Squares (BLS) to estimate the transit epoch, period, and
duration \citep{kovacs_box-fitting_2002}.  From the BLS parameters, we
isolated each transit to within $\pm$10 transit durations of its
midtime.  In each transit window, we then simultaneously fit for a model
transit, plus a local linear trend.  Our transit model uses the analytic
formulae calculated by \citet{mandel_analytic_2002} and implemented by
\citet{kreidberg_batman_2015}.

We allowed four free parameters per transit: the
time of mid-transit $t_{\rm tra}$, the planet-to-star radius ratio
$R_{\rm p}/R_\star$, and the slope and intercept of the line.
% Our priors were wide uniform distributions centered on the BLS
% estimates for $t_{\rm tra}$ and $R_{\rm p}/R_\star$, and on a slope of
% zero and an intercept of unity for the line.
We fixed the remaining transit parameters as follows.  We set the
eccentricity to zero, and the longitude of periastron to $\pi/2$.  We
assumed a quadratic limb-darkening law, with coefficients interpolated
from the \citet{claret_limb_2017} tables.  For the period, $a/R_\star$,
and inclination, we adopted the values from~\citet{petrucci_no_2013}.
These are in reasonable agreement with the parameters reported by
\citet{gillon_improved_2009}, \citet{southworth_high-precision_2009},
and \citet{huitson_gemini_2017}.

Having defined the free and fixed parameters, we calculated an initial
guess for the free parameters by maximizing the likelihood.  We then
sampled over the posterior using the algorithm proposed by
\citet{goodman_ensemble_2010} and implemented by
\citet{foreman-mackey_emcee_2013}.  During this process, we used the
\texttt{corner} software to check that our posteriors were being
well-sampled~\citep{corner_2016}.

After performing the initial fit, we set the photometric error bars to
be equal to the standard deviation of the in-transit points of the
residual lightcurve.  We then reperformed the same fitting procedure,
using the empirically determined photometric errors.

% TODO: final numbers
To check that the measured uncertainties are plausible, we computed the
reduced $\chi^2$ for a linear ephemeris fit to the measured TESS transit
times.  We found that $\chi^2 = 8.4$, with 16 degrees of freedom.
Examining the residuals by eye showed that the error variance was being
overestimated, so we multiplied the measured TESS errors by a factor
$f=0.73$, forcing a reduced $\chi^2$ of unity.  This reduced the mean
transit time uncertainty from $\sim$30 seconds per transit to $\sim$22
seconds per transit.

Figure~\ref{fig:lightcurves} shows the light curves, best-fit transit
models, and residuals.  Table~1 reports the mid-transit times and
their uncertainties, and also includes the previously reported times
analyzed in \S~\ref{sec:timing}.

Taking the mean and standard deviation of the measured planet to star
radius ratio, we find $R_{\rm p}/R_\star=0.1538\pm0.0013$, in
agreement with previous determinations
\citep{wilson_wasp-4b_2008,gillon_improved_2009,winn_transit_2009,southworth_high-precision_2009}.

Binning the residuals to 1-hour windows and taking the median absolute
deviation, we measure a MAD of the residual lightcurves of $594\,{\rm
ppm}\,{\rm hr}^{1/2}$.  WASP-4 has a TESS-band magnitude of $T=11.78$
\citep{stassun_TIC_2018}, corresponding to a predicted photon-counting
noise\footnote{\url{github.com/lgbouma/tnm}} of $410\,{\rm ppm}\,{\rm
hr}^{1/2}$.  At $T=11.78$, with a 9-pixel aperture read noise is
expected to contribute an additional $202\,{\rm ppm}\,{\rm hr}^{1/2}$ in
quadrature.  The predicted zodiacal background contribution is
$673\,{\rm ppm}\,{\rm hr}^{1/2}$, based on the
\citet{winn_photonflux_2013} model, which was used in the
\citet{Sullivan_2015} planet detection simulations.  The predicted level
of zodiacal noise seems to have been an overestimate, given the observed
RMS of WASP-4.  A more detailed assessment of TESS's photometric
performance is beyond the scope of this work.

\section{Timing analysis}
\label{sec:timing}

\subsection{Times}
\label{subsec:times}

\begin{figure*}[t]
    \begin{center}
        \leavevmode
        \includegraphics[width=0.9\textwidth]{f3.pdf}
    \end{center}
    \vspace{-0.5cm}
    \caption{
        {\bf Timing residuals and best-fit models for WASP-4b.}
        Points are differences between observed transit times and a
        linear ephemeris fit to all the times.  The constant period
        model is a poor fit to the observed times (gray line).  Black
        points are the more precise half of measured times; gray points
        are the less precise half.  The blue curve is the best-fit model
        with a constant period derivative.  The orange curve is the
        best-fit model assuming apsidal precession causes the observed
        timing deviations.
        \label{fig:times}
    }
\end{figure*}

Table~1 shows all the transit times used in our analysis.  We included
data from peer-reviewed literature for which the analysis was based on
observations of a single, complete transit, and for which the midpoint
was fit as a free parameter. We also required that the time system be
clearly documented.

Table~2 shows the occultation times.  Since there are fewer available,
we included all the occultation measurements we could find. The
tabulated values have been corrected for the light-travel time across
the diameter of the orbit by subtracting $2a/c = 22.8$ seconds from
the observed time.

We use the homogeneous \citet{hoyer_tramos_2013} timing study as our
starting point.  We took their BJD$_{\rm TT}$ times to be equivalent to
times in BJD$_{\rm TDB}$, since the two time systems differ periodically
by only milliseconds \citep{urban_explanatory_2012}.  We verified that
their times, as well as all others, were on the BJD$_{\rm TDB}$ time
standard using the calculator provided
by~\citet{eastman_achieving_2010}.  We set the zero-point epoch to be as
close as possible to the average of the transit times, weighted as the
inverse midtime variance. This minimizes the covariance between the
transit epoch and the period \citep{gibson_gemini_2013}.

Important contributions to the timing dataset are as follows.  The
earliest epoch in our fit is from the EulerCam lightcurve observed by
\citet{wilson_wasp-4b_2008}.  The second, more precise, epoch comes from
$z$-band photometry acquired by \citet{gillon_improved_2009} at the VLT
with FORS2.  Subsequent epochs came from
\citet{winn_transit_2009}\footnote{\citet{winn_transit_2009} recorded
the GPS clock time in UT reference. They then converted this to BJD in
the TDB reference for their reported midtimes.},
\citet{dragomir_terms_2011}, \citet{sanchis-ojeda_starspots_2011}, and
\citet{nikolov_wasp-4b_2012}.
%\citet{winn_transit_2009} measured two transits with the Baade 6.5-m
%at Las Campanas\footnote{\citet{winn_transit_2009} recorded the GPS
%clock time in UT reference. They then converted this to BJD in the
%TDB reference for their reported midtimes.}, and reported transit
%midtimes precise to 6~seconds.  \citet{dragomir_terms_2011} updated
%the ephemeris using two transits measured with the CTIO 0.9-m and
%1.0-m.  \citet{sanchis-ojeda_starspots_2011} reported four more
%transits from the Baade telescope at Las Campanas\footnote{
%\citet{sanchis-ojeda_starspots_2011} also used starspot occultations
%to show that the star's rotation axis is nearly aligned with the
%planet's orbital axis, in agreement with the Rossiter-McLaughlin
%measurement by \citet{triaud_spin-orbit_2010}.}.
%\citet{nikolov_wasp-4b_2012} observed 3 transits, in 4 independent
%photometry bands (Sloan {\it g'}, {\it r'}, {\it i'}, {\it z'}) with
%the MPG/EOS-2.2~m at La Silla.
\citet{hoyer_tramos_2013} reported nine transit observations taken with
the the Y4KCAM on the SMARTS 1-m telescope, and another three with the
SOI on the SOAR 4.2-m telescope.  \citet{hoyer_tramos_2013} also
homogeneously redetermined the transit times from all previous studies.
Though their latest two epochs fell slightly below the prediction of a
linear ephemeris (Panel C of their Figure~6), they did not find a need
to fit a quadratic function to the O-C values.

After the study by \citet{hoyer_tramos_2013},
\citet{ranjan_atmospheric_2014} acquired near-IR spectra of WASP-4b
using HST's WFC3 in both transit and secondary eclipse.  We adopt only
their transit time, since the epoch of occultation was not fit as a free
parameter.  \citet{huitson_gemini_2017} acquired optical transmission
spectra with the 8.1-m Gemini South telescope over 2011 to 2014, one
transit per season.  The per-point standard deviation of their
lightcurves is a few hundred parts per million, less than a factor of 3
away from the photon noise limit.  The uncertainties on their reported
transit times~--~on average 5.6~seconds~--~ are small, but possible
given the quality of their data~\citep[cf.][]{carter_analytic_2008}.

There have been other studies of WASP-4b for which the transit timing
data are either hetereogenous, not available, or not useable.  We
omitted the timing measurements from
\citet{southworth_high-precision_2009}, since there were technical
problems with the computer clock at the time of
observation~\citep{nikolov_wasp-4b_2012}.  The timing study
by~\citet{petrucci_no_2013} ruled out transit timing variations larger
than 54~seconds over the previous 5 years, but did not report transit
times.  \citet{baluev_benchmarking_2015} collected transit times for
many hot Jupiters, including WASP-4b, and included times from the
Exoplanet Transit Database
(ETD)\footnote{\url{http://var2.astro.cz/ETD/}}
\citep{poddany_ETD_2010}.  Since the ETD data come from heterogeneous
sources, their timestamps are less clearly documented, and the times
are thus more prone to systematic errors, we omit them from
consideration.
%We consider the effects of including ETD data in
%Appendix~\ref{sec:verify_archival_times}.
\citet{may_mopss_2018} obtained optical spectra using IMCAS on
Magellan, and found a flat transmission spectrum, in agreement with
the results reported by \citet{huitson_gemini_2017}.  They did not
report transit times.

There are fewer available occultation times.
\citet{beerer_secondary_2011} observed two occultations of WASP-4b using
warm Spitzer in the 3.6\,$\mu$m and 4.5\,$\mu$m bands.
\citet{caceres_ground-based_2011} detected an occultation from the
ground in the ${\rm K}_{\rm s}$ band, and gave a time in HJD, without
specifying the time standard.  We correspondingly added $69.184$ seconds
of uncertainty to their reported errors, in quadrature.  Finally, as
previously mentioned, \citet{ranjan_atmospheric_2014} acquired
occultation data with HST but kept the epoch as a fixed parameter.
\citet{zhou_secondary_2015} performed a similar analysis with
occultation data from the Anglo-Australian Telescope, focusing on the
eclipse depth, but also fitting for $e\cos\omega$.  We converted their
$e\cos\omega$ results into a midtime, using their ephemeris and the
standard formula \citep[{\it
e.g.},][]{charbonneau_detection_2005,winn_exoplanet_2010}
\begin{equation}
  t_{\rm occ}(E) =
  t_0 +  P E  +
  \frac{P}{2} \left( 1 + \frac{4}{\pi} e\cos\omega \right).
  \label{eq:occultation_time}
\end{equation}
This gives us a total of 4 occultations.  Although their timing error
bars are large compared to the transits, we include them for
completeness, and because they can help in our modeling of the timing
variations.

\subsection{Analysis}

After collecting the literature times and measuring the TESS times, we
compared the observed time to the prediction based solely on the
literature times. The result is shown in
Figure~\ref{fig:arrived_early}. The TESS observations appear to have
arrived early.  One immediate concern was systematic errors in either
the archival or the TESS timestamps. The tests we used to mitigate the
latter possibility are described in Appendix~\ref{sec:verify_tess}.

Assuming that the observed timing variation is astrophysical, we
proceeded by exploring three models for the timing data, identical to
the study by~\citet{patra_2017}.  

The first model assumes a constant orbital period on a circular orbit:
\begin{align}
  t_{\rm tra}(E) &= t_0 + PE,\\
  t_{\rm occ}(E) &= t_0 + \frac{P}{2} + PE,
\end{align}
for $E$ the epoch number.
The two free parameters are the reference epoch $t_0$ and the period $P$.

The second model assumes a constant period derivative, and a circular
orbit:
\begin{align}
  t_{\rm tra}(E) &=
    t_0 + PE +
    \frac{1}{2} \frac{{\rm d}P}{{\rm d}E} E^2, \\
  t_{\rm occ}(E) &=
    t_0 + \frac{P}{2} + PE +
    \frac{1}{2} \frac{{\rm d}P}{{\rm d}E} E^2.
\end{align}
The three free parameters are the epoch, period at the reference epoch,
and period derivative, ${\rm d}P/{\rm d}t = (1/P) {\rm d}P/{\rm d}E$. 

The third model assumes the planet has a slightly eccentric orbit, and
that the line of apsides is rotating:
\begin{align}
  t_{\rm tra}(E) &= 
		t_0 + P_{\rm s}E
    - \frac{e P_{\rm a}}{\pi} \cos\omega,\\
  t_{\rm occ}(E) &= 
    t_0 + \frac{P_{\rm a}}{2} + P_{\rm s}E
    + \frac{e P_{\rm a}}{\pi} \cos\omega,
\end{align}
for $P_{\rm s}$ the sidereal period, $e$ the eccentricity, $P_{\rm a}$
the anomalistic period, and $\omega$ the argument of pericenter.
Following~\citet{gimenez_revision_1995} and~\citet{patra_2017}, in
this model the angular velocity of the line of apsides ${\rm
d}\omega/{\rm d}E$ is constant,
\begin{equation}
  \omega(E) = \omega_0 + \frac{{\rm d}\omega}{{\rm d}E} E,
\end{equation}
and the sidereal and anomalistic periods are related by
\begin{equation}
  P_{\rm s} = P_{\rm a} \left(
    1 - \frac{1}{2\pi}\frac{{\rm d}\omega}{{\rm d}E}
    \right).
\end{equation}
The five free parameters are the epoch, sidereal period, eccentricity,
argument of pericenter at the reference epoch, and angular velocity of
line of apsides:
$(t_0, P_{\rm s}, e, \omega_0, {\rm d}\omega/{\rm d}E)$.

Assuming the error bars on the transit midtimes are gaussian and
independent, we then proceed by fitting each model to the timing data.
For the linear model, we took uniform priors over small windows around
the expected parameters.  For the quadratic model we did the same, and
took the quadratic term to have a uniform prior corresponding to
$
  Q_\star' \sim \mathcal{U}[10^3, 10^9],
$
for $Q_\star'$ the modified quality factor defined in
\S~\ref{sec:implications}.  For the precession model, we took the same
priors in epoch and sidereal period, and drew the remaining parameters
from wide uniform distributions.

%TODO: final numbers
Figure~\ref{fig:times} shows the residuals with respect to the linear
model.  The median MCMC linear fit has $\chi^2 = 149$ and 58 degrees of
freedom.  The median MCMC quadratic fit has $\chi^2 = 48.0$ and 57
degrees of freedom.  The median MCMC precession fit has $\chi^2 = 51.0$
and 55 degrees of freedom.

%TODO: final numbers
The difference in $\chi^2$ between the linear and quadratic fit
corresponds to $p \approx 10^{-23}$. In other words, assuming the linear
model is true, there is a $10^{-23}$ probability that simply by chance
we would see data that favors the quadratic model more strongly than the
observations.  A more likely possibility is some systematic error being
present in one of the most important sets of times: either the earliest
times, the Huitson et al.\ Gemini times (epoch numbers 44, 322, 591, and
851), or the TESS times.  We have taken every precaution against
systematic offsets in the TESS dataset (Appendix~\ref{sec:verify_tess}),
and Huitson et al.\ stand by their reported times (Huitson, priv.
comm.).  Since the linear model provides a poor fit to the data, we
discard it from further consideration.

%TODO: final numbers
The quadratic model fits better than the precession model.  It is
favored over the precession fit by $\Delta \chi^2 = 3.0$, and has two
fewer free parameters.  Another useful heuristic for model comparison is
the Bayesian Information Criterion (BIC),
\begin{equation}
  {\rm BIC} = \chi^2 + k\log n,
\end{equation}
for $k$ the number of free parameters, and $n$ the number of data
points. For us, $n=60$.  The difference in the BIC for the precession
and decay models is $\Delta {\rm BIC} = {\rm BIC}_{\rm prec} - {\rm
BIC}_{\rm quad} = 11$, corrsonding to a Bayes factor of $\approx
4\times10^{4}$.  In the language of~\citet{kass_bayes_1995}, this is
``decisive evidence'' for the quadratic model describing the data better
than the precession model.  Using the Akaike Information Criterion
similarly favors the orbital decay model, by $\Delta {\rm AIC} = 7$.

%TODO: final numbers
Assuming the orbital decay model, we find a period derivative 
\begin{equation}
\dot{P}
    = - (3.82^{+0.38}_{-0.37})\times 10^{-10}
    = - (12.1^{+1.2}_{-1.2})\,{\rm ms}\,{\rm yr}^{-1}.
\end{equation}
For comparison, the period derivative seen by
\citet{maciejewski_departure_2016} and \citet{patra_2017} in WASP-12b is
$\dot{P} = -29 \pm 3\,{\rm ms}\,{\rm yr}^{-1}$.  Assuming the quadratic
model, WASP-4b is thus decaying at about half the rate as WASP-12b.

%TODO: final numbers
If we instead assume the precession model, the best-fit
eccentricity is $e = (1.63^{+ 1.48}_{- 0.65})\times10^{-3}$. The
longitude of periapse correspondingly advances by $\dot{\omega}
= 14.8^{+5.4}_{3.8}\ {\rm degrees}\,{\rm yr}^{-1}$.   The
corresponding precession period would be $24^{+8}_{-6}$ years.

The median parameters and their standard deviations are reported for all
the models in Table~3.  To summarize, a linear ephemeris does not
describe the observed times.  Both orbital decay and precession can fit
the observed data.  However, the precession model has two extra free
parameters, and is statistically disfavored.  Further observations will
help discriminate between orbital decay and precession.

\section{Implications}
\label{sec:implications}

\begin{figure}[t!]
  \begin{center}
    \includegraphics[width=0.415\textwidth]{f4.pdf}
  \end{center}
  \vspace{-0.5cm}
  \caption{
    {\bf WASP-4b has a short eccentricity
    	damping time and medium decay time compared to other hot Jupiters.
    	It seems to orbit a main sequence host.
    }
    {\bf \it Top:}
    Transiting giant planets from \citet{bonomo_gaps_2017}, who
    measured the eccentricities using radial velocities.  Solid
    squares (including WASP-18b) have significant eccentricity
    detections.  The axes are chosen so that constant eccentricity
    damping timescales (Equation~\ref{eq:de_dt}) are lines.  The
    contours shown assume $Q_{\rm p}' = 10^5$.
    {\bf \it Middle:} 
    Axes are chosen so to that constant orbital decay timescales
    (Equation~\ref{eq:da_dt}) are lines.  The contours shown assume
    $Q_\star' = 10^7$.
    {\bf \it Bottom:}
    HR diagram highlighting WASP-12b, which shows timing variations
    similar to WASP-4b.  We also show hot Jupiters for which we
    detect no timing variation (see \S~\ref{sec:hj_verification}).
    \label{fig:context}
  }
\end{figure}

 
\subsection{Orbital decay}

The middle panel of Figure~\ref{fig:context} shows the expected
orbital decay time of WASP-4b compared to other transiting giant
planets.  Though there are roughly 20 hot Jupiters that might be
expected to decay faster, some orbit stars with minimal convective
zones ({\it e.g.}, WASP-18), and most are not as well-observed as
WASP-4.

Let us assume for the sake of argument that the observed timing
variation is caused entirely by orbital decay.  If the period
decreases at a fixed rate, it will reach zero after
\begin{equation}
  \frac{P}{ \dot{P} } = 9.6 \, {\rm Myr}.
\end{equation}
For comparison, the ``survival time'' found by \citet{patra_2017} for
WASP-12b was 3.2\,Myr.

Let us assume further that the ``constant phase lag'' model for tidal
interaction in a binary applies \citep{zahn_tidal_1977}.  The fluid
response is parametrized by a constant modified\footnote{For stars,
$k_\star \sim \mathcal{O}(10^{-2})$, so it is important to explicitly
distinguish $Q_\star'$ from $Q_\star$ \citep[{\it
e.g.},][]{schwarzschild_structure_1958}.} quality factor, $Q_\star' =
3 Q_\star / (2k_\star)$.  Here $k_\star$ is the stellar Love number,
which is smaller when the star's density distribution is more
centrally concentrated. $Q_\star$ is the ratio of the energy stored in
the equilibrium deformation of the star, divided by the energy
dissipated per tidal period \citep[{\it e.g.},][]{goldreich_q_1966}.
A larger $Q_\star$ implies less efficient dissipiation.  Though the
fluid mechanics responsible for the dissipation should depend on the
amplitude and frequency of the tidal perturbation
\citep[][Section~3.3]{ogilvie_tidal_2014}, we will treat $Q_\star'$ as
a constant.  This should not be taken literally; $Q_\star'$ may in
fact change sharply with varying tidal frequency
\citep{penev_empirical_2018}.

If the planet's spin and orbit are synchronized, the star is slowly
rotating, and the eccentricity is small, then the semi-major axis and
eccentricity evolve as
\citep[][Appendix B]{metzger_optical_2012}
\begin{align}
  \frac{1}{\tau_{\rm e}} &=
  \frac{|\dot{e}|}{e} =
    \frac{63 \pi } {2 Q_{\rm p}' }
    \left( \frac{R_{\rm p}}{a} \right)^5
    \left( \frac{M_\star}{M_{\rm p}} \right)
    \left( \frac{1}{P} \right)
  \label{eq:de_dt}
  \\
  \frac{1}{\tau_{\rm a}} &=
  \frac{|\dot{a}|}{a} =
    \frac{9 \pi } {Q_\star' }
    \left( \frac{R_\star}{a} \right)^5
    \left( \frac{M_{\rm p}}{M_\star} \right)
    \left( \frac{1}{P} \right).
  \label{eq:da_dt}
\end{align}
The orbital period evolves as
\begin{equation}
\label{eq:dP_dt}
  \dot{P} = -\frac{27\pi}{2 Q_\star'}
            \left(\frac{M_{\rm p}}{M_\star}\right)
            \left(\frac{R_\star}{a}\right)^5.
\end{equation}
Inserting our observed value of $\dot{P}$, the current dissipation
rate corresponds to a modified quality factor for WASP-4 of
\begin{equation}
	Q_\star' \approx 3.3\times10^4. 
\end{equation}
This is rather small.  Jupiter has $Q_{\rm Jup}' \approx 1.4 \times
10^5$, based on the motion of the Galilean moons
\citep{lainey_strong_2009}.  For stars, different $Q_\star'$ values have
been estimated in different contexts.  Studies of the observed
eccentricity distribution in stellar binaries have yielded $10^5 \gtrsim
Q_\star' \gtrsim 10^{7}$ \citep[{\it
e.g.},][]{meibom_robust_2005,belczynski_compact_2008,
geller_direct_2013,milliman_wiyn_2014}.  However the dissipation rates
may be different in hot Jupiter systems, since the tidal forcing
frequencies and amplitudes are different.  In fact, for the observed
population of hot Jupiters, \citet{penev_constraining_2012} argued that
$Q_\star' > 10^7$ would be required at some phase of hot Jupiter
evolution at the 99\% level, or else the observed population would not
exist (though \citealt{birkby_wts-2_2014} point out more efficient
dissipation would be allowed if the hot Jupiters arrived later).
Developing a generative model for the orbital separation distribution of
the hot Jupiter population, \citet{cameron_hierarchical_2018} similarly
found $10^7 \gtrsim Q_\star' \gtrsim 10^8$.

Using an empirical approach, \citet{penev_empirical_2018} reported a
$Q_\star'$ measurement for WASP-4.  In earlier work,
\citet{penev_hats-18b_2016} noted that HATS-18b and WASP-19b orbited
host stars whose isochronal ages are consistent with the Sun, but
whose stellar rotation periods are about 10 days
(\citealt{pont_empirical_2009} pointed out similar evidence for hot
Jupiter host spin-up).  The expected rotation periods at solar ages
are 20 to 30 days
(\citealt{schatzman_theory_1962,skumanich_time_1972}, and measurements
by \citealt{barnes_rotation_2016} in the 4\,Gyr-old M67 cluster).
Assuming that the difference between the observed and predicted
stellar rotation period was due to tides depositing angular momentum
in the star, \citet{penev_empirical_2018} modeled the evolution of hot
Jupiter systems under the influence of a magnetized wind and a
constant phase-lag tide.  They found two-sided limits for $Q_\star'$
in 35 of the 188 systems they examined.  In these 35 systems, they
found that $Q_\star'$ varied from $10^5$ at tidal periods of 2 days to
$10^7$ at tidal periods of 0.5 days.  For WASP-4 specifically, their
method gave $Q_\star' \approx (1.2^{+1.0}_{-0.5})\times10^7$. This is
more than 2 orders of magnitude less efficient than the dissipation
rate inferred from the decaying orbit would imply.

A different model for tidal dissipation was discussed by
\citet{essick_orbital_2016}, who considered the nonlinear response of
the star to a coupled network of gravity modes excited at the base of
the convective zone, and braking near the stellar core.  In their
approach, the stellar tidal quality factors in hot Jupiter systems
vary from $Q_\star' \approx 10^5 - 10^6$.  Using their Equation 26,
the prediction for WASP-4 is that it has $Q_\star' = 7\times10^5$.
This is one order of magnitude larger than would be implied by the
observed period change.

The applicability of the \citet{essick_orbital_2016} model depends on
the evolutionary state of the star.  The bottom panel of
Figure~\ref{fig:context} shows WASP-4 in the context of other giant
planet hosts from the \citet{bonomo_gaps_2017} sample.  On the y-axis
is $G=g-\mu$, for $g$ the apparent Gaia-band magnitude, and $\mu$ the
distance modulus reported by \citet{gaia_collaboration_gaia_2018}.
The x-axis is the effective temperature from \citet{bonomo_gaps_2017},
which for WASP-4 agrees within $1\sigma$ of that reported by
\citet{petrucci_no_2013}.  By visual inspection of the HR diagram,
WASP-4 shows litle evidence of being evolved.  By a similar argument
though, WASP-12 would not clearly stand out as a candidate subgiant,
and \citet{weinberg_tidal_2017} suggest it may in fact be a subgiant
(though \citealt{bailey_understanding_2019} do not find evidence that
supports this suggestion).  A more thorough isochronal analysis would
be of interest, particularly if the transit and occultation times
continue to vary.  Such an analysis is beyond the scope of this study.

In short, if the observed period change is caused entirely by tidal
decay, it would imply a tidal dissipation efficiency stronger than
typical orbit-averaged predictions.  It might be possible that we are
observing at a special time, when the planet is near resonance with the
some stellar oscillation mode, and the energy dissipation is
exceptionally large \citep{ogilvie_tidal_2014,essick_orbital_2016}.
Tidal dissipation rates might also be increased if the star is turning
off the main sequence.  WASP-4 shows little evidence for this; it is
unclear that tidal decay could explain its changing period.


\subsection{Apsidal precession}
\label{sec:apsidal_precession}

Let us instead assume that the entire timing variation can be explained
as being a portion of an apsidal precession cycle.  The required orbital
eccentricity is then
\begin{equation}
  e = (1.6^{+1.5}_{-0.7})\times10^{-3},
\end{equation}
and the line of apsides is advancing by
\begin{equation}
  \dot{\omega} = 14.8^{+5.4}_{-3.8}\ {\rm degrees}\,{\rm yr}^{-1}.
\end{equation}
One full precession period would take about 24 years, or twice the
current observing baseline.

\citet{ragozzine_probing_2009} calculated apsidal precession periods for
hot Jupiters ranging between of order 10 and 100 years.  They
highlighted that for many hot Jupiters, including WASP-4b, over 90\% of
the apsidal precession comes from the tidal bulge of the planet.
Precession from general relativity, the planet's rotational bulge, and
the star's rotational and tidal bulges are minor contributors.
\citet{ragozzine_probing_2009} also calcuated that WASP-4b was expected
to have a relatively short precession period, of 120 years.

\citet{ragozzine_probing_2009} also noted that if the planet has
non-zero eccentricity, a measurement of $\dot{\omega}$ can be converted
into a measurement of the planetary Love number, $k_{2,\rm p}$.
From \citet{ragozzine_probing_2009} Equation 14, the implied
planetary Love number for WASP-4b is
\begin{equation}
k_{2,{\rm p}} = 1.6^{+2.1}_{-1.2}.
\end{equation}
For comparison, $k_{2,{\rm Jup}} \approx 0.59$
(\citealt{wahl_tidal_2016}, bug see \citealt{ni_empirical_2018}).  A
uniform density sphere has $k_2 = 1.5$.  Our reported uncertainties on
$k_{2,{\rm p}}$ are large because the eccentricity, reference time, and
${\rm d}\omega/{\rm d}E$ are quite correlated, and the measured
occultation times only barely constrain them (Figure~\ref{fig:future}).

\subsubsection{Empirical constraints on small eccentricities in hot
Jupiters}

Could the eccentricity truly be non-zero?  For WASP-4b, the eccentricity
damping timescale (Equation~\ref{eq:de_dt}) is $\tau_e = 0.29 (Q_{\rm
p}'/10^5) \, {\rm Myr}$.  The system appears to be well over $1\,{\rm
Gyr}$ \citep{winn_transit_2009} old, so any initial eccentricity should
have been erased.

The top panel of Figure~\ref{fig:context} compares the expected
eccentricity damping time of WASP-4b with other transiting giant
planets.  WASP-4b has one of the shortest known eccentricity damping
times.  The hot Jupiter with the nearest $\tau_e$ and a significant
measured eccentricity is WASP-18b, with $e = 0.0076 \pm 0.0010$
\citep{triaud_spin-orbit_2010,bonomo_gaps_2017}.  This is the smallest
significant eccentricity measured by the RV method, with the
next-smallest cases being HAT-P-20b ($e =
0.01580^{+0.0023}_{-0.0018}$), and WASP-38b ($e =
0.02780^{+0.003}_{-0.0028}$).  For the 231 planets studied by
\citet{bonomo_gaps_2017}, 2 (13) had $2\sigma$ upper limits of $e <
0.01$ ($e<0.02$).  For WASP-4b, the strongest $2\sigma$ upper-limit on
its eccentricity from RVs is $e<0.011$
\citep{husnoo_observational_2012,bonomo_gaps_2017}.  

Occultation timing can provide stronger eccentricity limits than RV
measurements, but they often come with caveats.  For WASP-4b, the
strongest limit on its eccentricity comes from
\citet{beerer_secondary_2011}, who observed occultations and derived a
$2\sigma$ upper limit on $| e\cos\omega | $ of 0.0024.
% \citet{caceres_ground-based_2011} report $e\cos\omega = 0.0027 \pm
% 0.0018$.  \citet{zhou_secondary_2015} report a weaker $e\cos\omega =
% -0.001 \pm 0.003$. 
Placing eccentricity limits stronger than $e< 10^{-3}$ has only been
possible for a few hot Jupiters, and with numerous false-starts.
HD 189733b, for example, was initially reported to have a non-zero
eccentricty, $e\cos\omega = 0.0010 \pm 0.0002$
\citep{knutson_map_2007}.  This result was retracted by
\citet{agol_climate_2010}, who used a larger dataset of 7 Spitzer
eclipses and found a smaller average offset in the secondary eclipse
time than \citet{knutson_map_2007}.  \citet{agol_climate_2010} also
noted that a larger fraction of the planet's observed occultation time
offset could be explained by detailed modeling of the planetary
hot-spot, which can shift occultation times by up to a few
minutes \citep{williams_resolving_2006}.  Including an updated
hot-spot model, \citet{agol_climate_2010} reported a 2$\sigma$ limit
of $e\cos\omega < 0.00023$, perhaps the strongest eccentricity limit yet
for a hot Jupiter.  A similar saga occurred for HD 209458b, for which
\citet{winn_measurement_2005} combined Spitzer secondary eclipses with
RV data and reported a small finite eccentricity.  They suspected that
the data were consistent with zero eccentricity, and that the lower
bound on $e$ was an artifact of an imperfect limb-darkening model.
\citet{crossfield_spitzer_mips_2012} eventually found stronger limited:
$e\cos\omega < 0.0007$ at $2\sigma$.  The final example is WASP-12b, for
which the discovery RVs and photometry gave a small finite eccentricity
measurement \citep[][at $3\sigma$]{hebb_wasp-12b_2009}.  A prompt
ground-based occultation measurement by
\citet{lopez-morales_day-side_2010} concurred that a small eccentricity
was favored with weak statistical significance. Further follow-up found
occultation times consistent with circular orbits; $| e\cos\omega| <
0.0040 $ \citep{campo_orbit_2011,croll_near-infrared_2011}.

In summary, empirical constraints on the ellipticity of hot Jupiter
orbits at the $e \lesssim 10^{-3}$ level are present only for HD 189733b
and HD 209458b, from occultation timing.  Once the timing offsets are of
order tens of seconds, eccentricities from occultation timing may have
systematic uncertainties because of the hot-spot issue
\citep{williams_resolving_2006,agol_climate_2010}.  With the exception
of WASP-18b, radial velocity measurements have provided eccentricity
constraints only at the $e > 10^{-2}$ level.  Empirical cannot rule out
hot Jupiter eccentricities at the level needed to contextualize WASP-4.

\subsubsection{Theoretical expectations for small eccentricities in hot
Jupiters}

Lacking empirical data, we now turn to theoretical mechanisms that could
produce small eccentricities.

\paragraph{Neighboring companion}
\citet{mardling_long-term_2007} considered the long-term tidal evolution
of hot Jupiters with companions.  They pointed out that although the
early phases of the two-planet eccentricity evolution would occur
quickly, the final phase of the joint eccentricity evolution towards
circularity would often occur on timescales several orders of magnitude
longer than the circularization time (see their Figures~4 and 5).  The
companion in their model is coplanar, and can have a mass down to an
Earth-mass; the main requirement is that both the hot Jupiter and the
outer companion start on eccentric orbits.

Another mechanism to excite the hot Jupiter's eccentricity is the
Kozai-Lidov mechanism \citep{lidov_evolution_1962,kozai_secular_1962}.
In this case, the orbital plane of the outer companion, ``c'', would
need to be inclined relative to that of the hot Jupiter, ``b'', by at
least $\sin^{-1} \sqrt{2/5} \approx 39^\circ$.
\citet{bailey_understanding_2019} point out how the joint requirements
of Kozai-Lidov oscillations with a non-detection of an additional RV
signal can be combined to produce lower and upper limits on the
companion's mass.  For the Kozai-Lidov mechanism to operate, from
\citet{bailey_understanding_2019} Equation 20 we need
\begin{equation}
  M_{\rm c} > 7.7\,M_\oplus
  \times \left( \frac{a_{\rm c}}{a_{\rm b}} \right)^{3/2},
  \label{eq:kozai_bound}
\end{equation}
where $M_{\rm b,c}$ and $a_{\rm b,c}$ are the mass and semi-major axis
of WASP-4b and the hypothetical WASP-4c, and we have assumed WASP-4b's
Love number is $k_{2,{\rm b}}\approx 0.6$.  For the RV signal of the
companion to remain undetected, it would need to be in the residual
$({\rm O-C})_{\rm RV} = 15.2\,{\rm m s}^{-1}$ reported by
\citet{triaud_spin-orbit_2010}.  Again following
\citet{bailey_understanding_2019}, this implies
\begin{align}
  M_{\rm c} &<
  ({\rm O-C})_{\rm RV}
  \left( \frac{M_\star a_b}{G} \right)^{1/2}
  \left( \frac{a_c}{a_b} \right)^{1/2}
  f^{-1/2}
  \nonumber
  \\
  M_{\rm c} &< 
  24.2\,M_\oplus
  \times 
  \left( \frac{a_c}{a_b} \right)^{1/2}
  f^{-1/2},
  \label{eq:rv_bound}
\end{align}
for $f(e_{\rm c},\omega_{\rm c}, i_{\rm c}) \propto \sin^2 i_{\rm c}$
a geometric prefactor that depends on the argument of periastron
$\omega_{\rm c}$ and inclination $i_{\rm c}$ of the exterior companion
(\citealt{bailey_understanding_2019} Equation 23).  Since WASP-4b
transits along the line of sight, $f$ can be arbitrarily small, and
both of the preceding limits can be easily satisfied.  This
mechanism could produce the needed eccentricity. However it would be more
satifying if long-term radial velocity monitoring showed an additional
planet in the system.  The chances of one existing with a mass between
1-13\,$M_{\rm Jup}$ and orbital semi-major axis between 1-20\,AU is
about 50\%, and a linear trend is currently present at $\sim 2\sigma$
\citep{knutson_friends_2014}.

\paragraph{Fluctuations in the gravitational potential from convection}
A separate mechanism that might produce the eccentricity is simply
stellar convection \citep[][Section 7]{phinney_pulsars_1992}.  Phinney
presents this idea by analogy with a pendulum, which is never really
at rest. Instead, collisions with air molecules randomly perturb its
momentum and energy, and the equipartition theorem of statistical
mechanics tells us the average energy is $\langle E_{\rm pend} \rangle
= k T_{\rm air}$.  In hot Jupiter systems, the ``air molecules'' are
convective eddies, and their associated kinetic energy leads the orbit
to never have exactly zero eccentricity.  In particular, from
\citealt{phinney_pulsars_1992} Equation 7.33,
\begin{equation}
  \langle e^2 \rangle =
  \frac{ 2 \langle E_{\rm e} \rangle }{\mu n^2 a^2}
  = 6.8\times10^{-5}
  \frac{(L^2 R_{\rm conv}^2 M_{\rm conv}^2)^{1/3}}{\mu n^2 a^2},
\end{equation}
for $L$ the stellar luminosity, $R_{\rm conv}$ ($M_{\rm conv}$) the
width (mass) of the convective region,  $\mu$ the reduced mass, $n$
the orbital frequency, and $a$ the semi-major axis.  For WASP-4,
convection in the star is more important than convection in the
planet, and $\langle e^2 \rangle^{1/2} < 10^{-5}$.


\subsection{Applegate effect}
Some eclipsing binaries exhibit period modulations with amplitudes of
$\lesssim 0.05\,{\rm days}$ over timescales of decades \citep[{\it
e.g.},][]{soderhjelm_geometry_1980,hall_relation_1989}.
\citet{applegate_mechanism_1992} proposed a mechanism to explain these
modulations in which the internal structure of a magnetically active
star changes shape via cyclic exchange of angular momentum between the
inner and outer zones of the star.  This model could also apply to a
hot Jupiter orbiting a star with a convective zone.  The changing
gravitational quadrupole of the star would cause the orbit of the
planet to precess on the timescale of the stellar activity cycle ("the
dynamo timescale").  An essential difference between this process and
apsidal precession is that it is expected to be quasi-periodic
\citep[{\it e.g.},][Figure~12]{soderhjelm_geometry_1980}.  The transit
and occultation deviations would also have the same sign, while for
apsidal precession they have opposite signs.  For WASP-4,
\citet{watson_orbital_2010} estimated that the effect could produce
timing deviations of up to 15 seconds, depending on the modulation
period of the stellar dynamo, and the corresponding level of
differential surface shear.  This cannot explain the majority our
observed $78$ second variation.


\subsection{Other possible explanations}

Another mechanism to produce the observed period change would be if an
unseen companion in the WASP-4 system accelerated the star towards us.
This would produce a period derivative
\begin{equation}
	\dot{P} \approx \frac{\dot{v}_{\rm r} P}{c},
\end{equation}
for $\dot{v}_{\rm r}$ the time derivative of the radial velocity.
Recently, \citet{bonomo_gaps_2017} collected all available HARPS data
for the system, and reported no detectable slope at $3\sigma$.
However, \citet{knutson_friends_2014} collected 5 Keck HIRES radial
velocity measurements for WASP-4, and merged them with measurements
from CORALIE and HARPS reported by \citet{wilson_wasp-4b_2008},
\citet{pont_determining_2011}, and \citet{husnoo_observational_2012}.
Using these four datasets, \citet{knutson_friends_2014} found that
$\dot{\gamma} = -0.0099^{+0.0052}_{-0.0054}$, corresponding to a
$2\sigma$ limit on the acceleration towards us of $\dot{\gamma} > -
0.021 \,{\rm m}\,{\rm s}^{-1}\,{\rm day}^{-1}$.  This limits $\dot{P}$
to be less than $9\times 10^{-11}$ at $2\sigma$.  
%FIXME TODO final value
Our measured value is $\dot{P} = -(3.9^{+0.7}_{-0.5})\times 10^{-10}$.
We can thus rule out radial acceleration as the sole cause of the
observed period change.  However, it could cause up to one-quarter of
the observed period decrease.  Improved long-term radial velocity
monitoring with the same instruments will help clarify the situation,
particularly given the fact that short-term trends of roughly
$\dot{\gamma} \approx 2.8 \,{\rm m}\,{\rm s}^{-1}\,{\rm day}^{-1}$ are
seen over timescales of order 10 days in
WASP-4~\citep{husnoo_observational_2012}.  Given the known spottedness
of the system~\citep{sanchis-ojeda_starspots_2011}, these
short-timescale variations may also have a contribution from stellar
activity.

There are two other small effects worth briefly mentioning.  The first
is the \citet{shklovskii_possible_1970} effect, in which the star's
proper motion leads to a changing radial velocity.  This would be
apparent in the Doppler measurements described above.  The effect
contributes a change of order $\dot{P} \sim P\mu^2 d/ c \sim
6\times10^{-13}$.  The second effect, described by
\citet{rafikov_stellar_2009}, comes from the star's on-sky motion
altering our viewing angle, and leads to an observed apsidal
precession.  The corresponding change is $\dot{P} \sim (P\mu)^2/2\pi
\sim 10^{-21}$.

Many of these phenomena are likely occurring simultaneously.  From the
preceding discussion, we found that radial acceleration,
proper-motion, and the Applegate mechanism are unlikely to cause the
timing deviations.  That leaves tidal decay and apsidal precession as
the possible causes.



\section{Future prospects}
\label{sec:future}

\begin{figure}[!t]
	\begin{center}
		\leavevmode
		\includegraphics[width=0.48\textwidth]{f5.pdf}
	\end{center}
  \vspace{-0.5cm}
	\caption{
		{\bf Future evolution possibilies for WASP-4b.}
		Dots are as in Figure~\ref{fig:times}.
		Lines are 100 random samples from the posteriors of the apsidal
		precession model (orange), and the orbital decay model (blue).
		In an extended mission, TESS might re-observe WASP-4b in the early
		2020s. While useful to confirm that the period is
		in fact changing, precise long-term monitoring past 2025 will
		be needed to resolve the two models.
		\label{fig:future}
	}
\end{figure}

If TESS continues observing after its primary mission, it will observe
additional transits of WASP-4b in the early 2020s
(Figure~\ref{fig:future}).  High-precision ground-based transit
measurements could also be useful.  In order to rule between orbital
decay and apsidal precession, occultation measurements in the near term,
and also in the mid-2020s, will be helpful.  Additionally, confirmation
and extension of the linear trend suggested by
\citet{knutson_friends_2014} would help in understanding whether an
exterior companion in the system might explain any residual
eccentricity.  These observations should be carried out, because they
will lead to new knowledge; either a measure of the efficiency of tidal
dissipation within the star, or the eccentricity and Love number of the
planet.  Measuring either for a single system would be rather
interesting.  In the coming decade, wide-field photometric surveys
(TESS, HATPI, NGTS, PLATO) might even enable such measurements at a
population level, which would be a good deal more interesting.


%Detailed modeling of the star's interior
% structure and and tidal evolution is warranted, as are continued
% observations of transits and occultations.
% 

\acknowledgements{
L.G.B.\ acknowledges helpful discussions with F. Dai and V. Van Eylen. 
%
We acknowledge the use of TESS Alert data, which is currently in a beta test
phase, from the TESS Science Office. Funding for the TESS mission is provided
by NASA's Science Mission directorate. This research has made use of the NASA
Exoplanet Archive, which is operated by the California Institute of Technology,
under contract with the National Aeronautics and Space Administration under the
Exoplanet Exploration Program.  This work made use of NASA's Astrophysics Data
System Bibliographic Services.
This research has made use of the VizieR catalogue access tool, CDS,
Strasbourg, France. The original description of the VizieR service was
published in A\&AS 143, 23.
This work has made use of data from the European Space Agency (ESA) mission
{\it Gaia} (\url{https://www.cosmos.esa.int/gaia}), processed by the {\it Gaia}
Data Processing and Analysis Consortium (DPAC,
\url{https://www.cosmos.esa.int/web/gaia/dpac/consortium}). Funding for the DPAC
has been provided by national institutions, in particular the institutions
participating in the {\it Gaia} Multilateral Agreement.
%
\newline
%
\facility{
	TESS \citep{ricker_transiting_2015},
	Gaia \citep{gaia_collaboration_gaia_2016,gaia_collaboration_gaia_2018}}
}
%
\software{
  \texttt{astrobase} \citep{bhatti_astrobase_2018},
  \texttt{astropy} \citep{the_astropy_collaboration_astropy_2018},
  \texttt{astroquery} \citep{astroquery_2018},
  \texttt{BATMAN} \citep{kreidberg_batman_2015},
  \texttt{corner} \citep{corner_2016},
  \texttt{emcee} \citep{foreman-mackey_emcee_2013},
  \texttt{IPython} \citep{perez_2007},
  \texttt{matplotlib} \citep{hunter_matplotlib_2007}, 
  \texttt{numpy} \citep{walt_numpy_2011}, 
  \texttt{pandas} \citep{mckinney-proc-scipy-2010},
  \texttt{scikit-learn} \citep{scikit-learn},
  \texttt{scipy} \citep{jones_scipy_2001}.
  }

\bibliographystyle{yahapj}                            
\bibliography{bibliography} 

%\clearpage

%\newpage
%% \begin{deluxetable}{} command tell LaTeX how many columns
%% there are and how to align them.
\startlongtable
\begin{deluxetable}{ccccc}
    
%% Keep a portrait orientation

%% Over-ride the default font size
%% Use Default (12pt)
\tabletypesize{\scriptsize}

%% Use \tablewidth{?pt} to over-ride the default table width.
%% If you are unhappy with the default look at the end of the
%% *.log file to see what the default was set at before adjusting
%% this value.

%% This is the title of the table.
\tablecaption{WASP-4b transit times, uncertainties, and references.}
\label{tab:transit_times}

%% This command over-rides LaTeX's natural table count
%% and replaces it with this number.  LaTeX will increment 
%% all other tables after this table based on this number
\tablenum{1}

%% The \tablehead gives provides the column headers.  It
%% is currently set up so that the column labels are on the
%% top line and the units surrounded by ()s are in the 
%% bottom line.  You may add more header information by writing
%% another line between these lines. For each column that requries
%% extra information be sure to include a \colhead{text} command
%% and remember to end any extra lines with \\ and include the 
%% correct number of &s.
\tablehead{
  \colhead{$t_{\rm tra}$ [BJD$_\mathrm{TDB}$]} &
  \colhead{$\sigma_{t_{\rm tra}}$ [days]} &
  \colhead{Epoch} & 
  \colhead{H13?} & 
  \colhead{Reference}
}

%% All data must appear between the \startdata and \enddata commands
% XXX pasted in from selected_transit_times.tex
\startdata
 2454368.59279 &      0.00033 &   -1059 &       1 &           \citet{wilson_wasp-4b_2008} \\
 2454396.69576 &      0.00012 &   -1038 &       1 &          \citet{gillon_improved_2009} \\
 2454697.79817 &      0.00009 &    -813 &       1 &             \citet{winn_transit_2009} \\
 2454701.81280 &      0.00022 &    -810 &       1 &             \citet{hoyer_tramos_2013} \\
 2454701.81303 &      0.00018 &    -810 &       1 &             \citet{hoyer_tramos_2013} \\
 2454705.82715 &      0.00029 &    -807 &       1 &             \citet{hoyer_tramos_2013} \\
 2454728.57767 &      0.00042 &    -790 &       1 &             \citet{hoyer_tramos_2013} \\
 2454732.59197 &      0.00050 &    -787 &       1 &             \citet{hoyer_tramos_2013} \\
 2454740.62125 &      0.00035 &    -781 &       1 &             \citet{hoyer_tramos_2013} \\
 2454748.65111 &      0.00007 &    -775 &       1 &             \citet{winn_transit_2009} \\
 2454752.66576 &      0.00069 &    -772 &       1 &           \citet{dragomir_terms_2011} \\
 2455041.72377 &      0.00018 &    -556 &       1 &             \citet{hoyer_tramos_2013} \\
 2455045.73853 &      0.00008 &    -553 &       1 &  \citet{sanchis-ojeda_starspots_2011} \\
 2455049.75325 &      0.00007 &    -550 &       1 &  \citet{sanchis-ojeda_starspots_2011} \\
 2455053.76774 &      0.00009 &    -547 &       1 &  \citet{sanchis-ojeda_starspots_2011} \\
 2455069.82661 &      0.00029 &    -535 &       1 &          \citet{nikolov_wasp-4b_2012} \\
 2455069.82670 &      0.00028 &    -535 &       1 &          \citet{nikolov_wasp-4b_2012} \\
 2455069.82617 &      0.00038 &    -535 &       1 &          \citet{nikolov_wasp-4b_2012} \\
 2455069.82676 &      0.00031 &    -535 &       1 &          \citet{nikolov_wasp-4b_2012} \\
 2455073.84128 &      0.00026 &    -532 &       1 &          \citet{nikolov_wasp-4b_2012} \\
 2455073.84108 &      0.00029 &    -532 &       1 &          \citet{nikolov_wasp-4b_2012} \\
 2455073.84111 &      0.00023 &    -532 &       1 &          \citet{nikolov_wasp-4b_2012} \\
 2455073.84114 &      0.00018 &    -532 &       1 &          \citet{nikolov_wasp-4b_2012} \\
 2455096.59148 &      0.00022 &    -515 &       1 &             \citet{hoyer_tramos_2013} \\
 2455100.60595 &      0.00012 &    -512 &       1 &  \citet{sanchis-ojeda_starspots_2011} \\
 2455112.64986 &      0.00039 &    -503 &       1 &          \citet{nikolov_wasp-4b_2012} \\
 2455112.65009 &      0.00033 &    -503 &       1 &          \citet{nikolov_wasp-4b_2012} \\
 2455112.65005 &      0.00031 &    -503 &       1 &          \citet{nikolov_wasp-4b_2012} \\
 2455112.65005 &      0.00049 &    -503 &       1 &          \citet{nikolov_wasp-4b_2012} \\
 2455132.72310 &      0.00041 &    -488 &       1 &             \citet{hoyer_tramos_2013} \\
 2455468.61943 &      0.00046 &    -237 &       1 &             \citet{hoyer_tramos_2013} \\
 2455526.16356 &      0.00008 &    -194 &       0 &       \citet{ranjan_atmospheric_2014} \\
 2455828.60375 &      0.00041 &      32 &       1 &             \citet{hoyer_tramos_2013} \\
 2455832.61815 &      0.00041 &      35 &       1 &             \citet{hoyer_tramos_2013} \\
 2455844.66287 &      0.00009 &      44 &       0 &           \citet{huitson_gemini_2017} \\
 2456216.69123 &      0.00006 &     322 &       0 &           \citet{huitson_gemini_2017} \\
 2456576.67556 &      0.00005 &     591 &       0 &           \citet{huitson_gemini_2017} \\
 2456924.61561 &      0.00006 &     851 &       0 &           \citet{huitson_gemini_2017} \\
 2458355.18490 &      0.00024 &    1920 &       0 &                             This work \\
 2458356.52251 &      0.00026 &    1921 &       0 &                             This work \\
 2458357.86101 &      0.00024 &    1922 &       0 &                             This work \\
 2458359.19951 &      0.00025 &    1923 &       0 &                             This work \\
 2458360.53708 &      0.00027 &    1924 &       0 &                             This work \\
 2458361.87539 &      0.00024 &    1925 &       0 &                             This work \\
 2458363.21412 &      0.00027 &    1926 &       0 &                             This work \\
 2458364.55192 &      0.00025 &    1927 &       0 &                             This work \\
 2458365.89064 &      0.00026 &    1928 &       0 &                             This work \\
 2458369.90503 &      0.00027 &    1931 &       0 &                             This work \\
 2458371.24297 &      0.00026 &    1932 &       0 &                             This work \\
 2458372.58136 &      0.00027 &    1933 &       0 &                             This work \\
 2458373.91982 &      0.00027 &    1934 &       0 &                             This work \\
 2458375.25801 &      0.00024 &    1935 &       0 &                             This work \\
 2458376.59621 &      0.00024 &    1936 &       0 &                             This work \\
 2458377.93443 &      0.00026 &    1937 &       0 &                             This work \\
 2458379.27317 &      0.00026 &    1938 &       0 &                             This work \\
 2458380.61097 &      0.00027 &    1939 &       0 &                             This work \\
\enddata

%% Include any \tablenotetext{key}{text}, \tablerefs{ref list},
%% or \tablecomments{text} between the \enddata and 
%% \end{deluxetable} commands

%% General table comment marker
\tablecomments{
    $t_{\rm tra}$ is the measured transit midtime, and $\sigma_{t_{\rm
    tra}}$ is its $1\sigma$ uncertainty.
    $\sigma_{t_0}$ was evaluated from the sampled posteriors by taking
    the maximum of the difference between the 84th percentile
    minus the median, and the median minus the 16th percentile.
    The ``Reference'' column refers to the work describing the
    original observations.
    The ``H13?'' column is 1 if the mid-time value was taken from 
    \citet{hoyer_tramos_2013}.  Otherwise, the mid-time
    came from the column listed in ``Reference''.
    The \citet{hoyer_tramos_2013} BJD$_{\rm TT}$ times are equal to
    BJD$_{\rm TDB}$ for our purposes \citep{urban_explanatory_2012}.
    We omitted the timing measurements from
    \citet{southworth_high-precision_2009}, since there were technical
    problems with the computer clock at the time of
    observation~\citep{nikolov_wasp-4b_2012}.
}
\end{deluxetable}


%\clearpage
%\newpage
%% \begin{deluxetable}{} command tell LaTeX how many columns
%% there are and how to align them.
\startlongtable
\begin{deluxetable}{cccc}
    
%% Keep a portrait orientation

%% Over-ride the default font size
%% Use Default (12pt)
\tabletypesize{\footnotesize}

%% Use \tablewidth{?pt} to over-ride the default table width.
%% If you are unhappy with the default look at the end of the
%% *.log file to see what the default was set at before adjusting
%% this value.

%% This is the title of the table.
\tablecaption{WASP-4b occultation times, uncertainties, and references.}
\label{tab:occultation_times}

%% This command over-rides LaTeX's natural table count
%% and replaces it with this number.  LaTeX will increment 
%% all other tables after this table based on this number
\tablenum{3}

%% The \tablehead gives provides the column headers.  It
%% is currently set up so that the column labels are on the
%% top line and the units surrounded by ()s are in the 
%% bottom line.  You may add more header information by writing
%% another line between these lines. For each column that requries
%% extra information be sure to include a \colhead{text} command
%% and remember to end any extra lines with \\ and include the 
%% correct number of &s.
\tablehead{
  \colhead{$t_{\rm occ}$ [BJD$_\mathrm{TDB}$]} &
  \colhead{$\sigma_{t_{\rm occ}}$ [days]} &
  \colhead{Epoch} & 
  \colhead{Reference}
}

%% All data must appear between the \startdata and \enddata commands
% XXX pasted in from selected_transit_times.tex
\startdata
 2455102.61210 &      0.00074 &    -511 &  \citet{caceres_ground-based_2011}\tablenotemark{a} \\
 2455172.20159 &      0.00130 &    -459 &      \citet{beerer_secondary_2011} \\
 2455174.87780 &      0.00087 &    -457 &      \citet{beerer_secondary_2011} \\
 2456907.88714 &      0.00290 &     838 &        \citet{zhou_secondary_2015}\tablenotemark{b} \\
\enddata

%% Include any \tablenotetext{key}{text}, \tablerefs{ref list},
%% or \tablecomments{text} between the \enddata and 
%% \end{deluxetable} commands

%% General table comment marker
\tablecomments{
	$t_{\rm occ}$ is the measured occultation midtime, minus the
	$2a/c=22.8$ second light travel time;
	$\sigma_{t_{\rm occ}}$ is the $1\sigma$ uncertainty on the occultation
	time.
}
\tablenotetext{a}{
\citet{caceres_ground-based_2011} reported this time in ``HJD'', with
an unspecified time standard. We assumed the time was originally in
${\rm HJD}_{\rm UTC}$, and converted to ${\rm BJD}_{\rm TDB}$ for the
tabulated time.
}
\tablenotetext{b}{
\citet{zhou_secondary_2015} fixed the epoch, and let $e\cos\omega$
float. Using the reported dates of observation, we converted their
$e\cos\omega$ values into an occultation time using
Equation~\ref{eq:occultation_time} of the text. 
}

\end{deluxetable}


%\clearpage
%\newpage
%\renewcommand{\arraystretch}{1.0}

\startlongtable
\begin{deluxetable}{lc}

\tabletypesize{\footnotesize}

\tablenum{4}

%\tablewidth{0pt}

\tablecaption{Best-fit transit timing model parameters.}
\label{tab:bestfit}

\tablehead{
  \colhead{Parameter} &
  \colhead{Median Value~(Unc.)\tablenotemark{a}}
}

\startdata
~~~~~~{\it Constant period} &  \\
$t_0$\,[${\rm BJD}_{\rm TBD}$]    & 2455804.515752(+19)(-19)              \\
$P$\,[days]                       & 1.338231466(+23)(-22)                 \\
~~~~~~{\it Constant period derivative} &  \\
$t_0$~[${\rm BJD}_{\rm TBD}$]     & 2455804.515918(+24)(-24)              \\
$P$\,[days]                       & 1.338231679(+31)(-31)                 \\
$dP/dt$                           & $-4.00(+37)(-38) \times 10^{-10}$     \\
~~~~~~{\it Apsidal precession (wide prior)} &  \\
$t_0$~[${\rm BJD}_{\rm TBD}$]     & 2455804.51530(+25)(-31)               \\
$P_{\rm s}$\,[days]               & 1.33823127(+20)(-48)                  \\
$e$                               & $1.92^{+1.93}_{-0.76} \times 10^{-3}$ \\
$\omega_0$\,[rad]                 & 2.40(+38)(-34)                        \\
$d\omega/dE$~[rad\,epoch$^{-1}$]  & $8.70^{+3.01}_{-2.30} \times 10^{-4}$ \\
\enddata
\tablenotetext{a}{
The numbers in parenthesis give the $68\%$ confidence interval for the final
two digits, where appropriate.
}
\end{deluxetable}


\clearpage
\newpage

\appendix

\section{Verifying the TESS Time Stamps}
\label{sec:verify_tess}

Any systematic offset between the reported TESS
times and the true barycentric reference would cast doubt on the
results of this study.  Such an offset would have historic precedent:
Kepler had a systematic error in its timestamps that was corrected
only in Q14~\citep[][Section 3.4]{kepler_DR19_2013}.

We devised two checks on the absolute calibration of the TESS time
system.  First, in \S~\ref{sec:headers}, we recalculate and confirm
the barycentric correction reported by SPOC in the TESS lightcurve
file headers.  In \S~\ref{sec:hj_verification}, we repeat the main
analysis of the paper for a collection of other hot Jupiters, and use
their observed transit times to rule out a global TESS time offset.
The latter test confirms that WASP-4b is an outlier.

\subsection{Using the headers of the lightcurve files}
\label{sec:headers}

The TESS lightcurve files provide observation start and end times in
three different time systems: ${\rm JD}_{\rm UTC}$, ${\rm JD}_{\rm
TDB}$, and ${\rm BJD}_{\rm TDB}$.
First, we verified for a few select lightcurve files that ${\rm
JD}_{\rm UTC}$ lagged behind ${\rm JD}_{\rm TDB}$ by the expected
$32.184 + N\,{\rm seconds}$, where $N$ is the number of leap-seconds
since 1961. For the relevant observation time, $N=37$, and the offset
was as expected.
Then, using the ${\rm JD}_{\rm UTC}$ timestamp, we recalculated the
barycentric correction computed by SPOC, using the
\citealt{eastman_achieving_2010} calculator.  Due to the first
author's ignorance of the spacecraft's position, we performed this
calculation assuming that the observer was located at the Earth's
geocenter, and used the correct direction for each star.  This gave us
times in ${\rm BJD}_{\rm TDB}$ that agreed with the archival times to
within 1.7 seconds.  This offset is comparable to the light-travel
delay time expected for TESS on its orbit from perigee of $\approx
20R_\oplus$ to apogee of $\approx 60R_\oplus$.

This calculation bounds any error in the SPOC barycentric julian date
correction to be less than 1.7 seconds.  Since this is smaller than
the effect of interest, we treat the BJD correction as ``verified'',
and proceed to a subsequent test.


\subsection{Using other hot Jupiters as references}
\label{sec:hj_verification}

\begin{figure*}[ht!]
  \begin{center}
    \leavevmode
    \includegraphics[width=0.9\textwidth]{f6.pdf}
  \end{center}
  \vspace{-0.5cm}
  \caption{
    {\bf There is no evidence for a systematic offset between TESS
    times and the barycentric reference.}
    While the WASP-4b transits fell about 75 seconds earlier than
    predicted, other well-observed hot Jupiters, in particular WASP-6b
    and WASP-18b, arrived on time.  Ticks are observed TESS transit
    midtimes; the blue distribution function is a kernel density
    estimate; the orange distribution function is a gaussian centered
    on zero using the indicated $1\sigma$ standard deviation in the
    prediction
    ($\sigma_{\rm predicted}$).
    \label{fig:hjs}
  }
\end{figure*}

If the observed timing delay in WASP-4b were caused by a systematic
timing system offset between the TESS ${\rm BJD}_{\rm TDB}$ times and
the ${\rm BJD}_{\rm TDB}$ reference, we would expect that it might
apply to other hot Jupiters as well.  This test can rule out an
important class of clock error~--~a systematic global offset.

To rule out this possibility, we repeat the timing analysis of the
main paper, for other hot Jupiters with long preceding observing
baselines.  We first checked which hot Jupiters were observed over the
first two TESS sectors using a combination of
\texttt{tessmaps}\footnote{\url{github.com/lgbouma/tessmaps}} and
TEPCat \citep{southworth_homogeneous_2011}.  We then selected hot
Jupiters for which there were at least five distinct epochs reported
in the peer-reviewed literature.  We required that each observation be
of a single transit, that the midpoint be fit as a free parameter, and
that the time system be clearly documented.  Our final sample of
hot Jupiters included WASPs-4b, -5b, -6b, -18b, and -46b.  The
collected and measured times are given in Tables~4, 5, 6, and 7
for each.

Using the literature timing data for each hot Jupiter, we then
performed a least-squares fit to a linear ephemeris.  Using the
best-fit values and variances, we calculated the uncertainty on the
predicted transit time during the TESS observations.  This gave 9, 94,
18, 42, and 60 seconds for WASPs-4b, -5b, -6b, -18b, and -46b.  If a
substantial portion of the observed timing deviation in WASP-4b (about
77 seconds) were from a systematic offset in the time systems, this
offset would be present in the other hot Jupiter transit times as
well.  We show in Figure~\ref{fig:hjs} that WASP-4b is the only hot
Jupiter that transited significantly earlier than expected.

To convert this intuition into a quantitative limit, for each hot
Jupiter we considered the model
\begin{equation}
  t_{\rm tra}(E) = t_0 + PE + t_{\rm offset},
\end{equation}
for $t_{\rm offset}$ a systematic constant offset between the reported
timestamps and the true ${\rm BJD}_{\rm TDB}$ reference.  Our priors
were
\begin{align}
  t_0 &\sim \mathcal{N}[t_0', \sigma_{t_0'}], \\
  P &\sim \mathcal{N}[P', \sigma_{P'}], \\
  t_{\rm offset} &\sim \mathcal{U}[-20\sigma_{t_0'},20\sigma_{t_0'}],
\end{align}
where $\mathcal{N}$ and $\mathcal{U}$ denote a normal and uniform
distribution, $(t_0', P')$ are the best-fit reference time and period
using only the literature transit times, and $(\sigma_{t_0'},
\sigma_{P'})$ are their uncertainties.

For each planet, we then ask: what fraction of the posterior for
$t_{\rm offset}$ is consistent with an offset worse than
$77$ seconds?  For WASP-4b, the answer is unsurpringly about half.
For WASP-6b, the most constraining object, about 1 sample in 2 million
is consistent with such a timing offset ($4.9\sigma$).  For WASP-18b,
1 in 63 samples would be consistent with this timing offset
($2.1\sigma$), and in WASP-46b, the limit is 1 in 38 samples
($1.9\sigma$).  For WASP-5b, the predicted time is too imprecise to
rule out timing offsets at the necessary amplitude.  Multiplying the
three independent probabilities for WASPs-5b, 6b, and -18b, we rule
out $t_{\rm offset} < -77\ {\rm seconds}$ at $6.3\sigma$, or about
about 1 part in 5 billion.


%\clearpage
%\newpage
%% \begin{deluxetable}{} command tell LaTeX how many columns
%% there are and how to align them.
\startlongtable
\begin{deluxetable}{ccccc}
    
%% Keep a portrait orientation

%% Over-ride the default font size
%% Use Default (12pt)
\tabletypesize{\scriptsize}

%% Use \tablewidth{?pt} to over-ride the default table width.
%% If you are unhappy with the default look at the end of the
%% *.log file to see what the default was set at before adjusting
%% this value.

%% This is the title of the table.
\tablecaption{WASP-5b transit times, uncertainties, and references.}
\label{tab:WASP-5b}

%% This command over-rides LaTeX's natural table count
%% and replaces it with this number.  LaTeX will increment 
%% all other tables after this table based on this number
\tablenum{4}

%% The \tablehead gives provides the column headers.  It
%% is currently set up so that the column labels are on the
%% top line and the units surrounded by ()s are in the 
%% bottom line.  You may add more header information by writing
%% another line between these lines. For each column that requries
%% extra information be sure to include a \colhead{text} command
%% and remember to end any extra lines with \\ and include the 
%% correct number of &s.
\tablehead{
  \colhead{$t_{\rm tra}$ [BJD$_\mathrm{TDB}$]} &
  \colhead{$\sigma_{t_{\rm tra}}$ [days]} &
  \colhead{Epoch} & 
  \colhead{Reference}
}

%% All data must appear between the \startdata and \enddata commands
\startdata
 2454383.76750 &      0.00040 &    -885 &           \citet{anderson_wasp-5b_2008} \\
 2454387.02275 &      0.00100 &    -883 &           \citet{anderson_wasp-5b_2008} \\
 2454636.17459 &      0.00082 &    -730 &         \citet{fukui_measurements_2011} \\
 2454699.68303 &      0.00041 &    -691 &              \citet{hoyer_transit_2012} \\
 2454707.82465 &      0.00052 &    -686 &              \citet{hoyer_transit_2012} \\
 2454707.82523 &      0.00025 &    -686 &  \citet{southworth_high-precision_2009} \\
 2454730.62243 &      0.00031 &    -672 &  \citet{southworth_high-precision_2009} \\
 2454730.62301 &      0.00076 &    -672 &              \citet{hoyer_transit_2012} \\
 2454761.56356 &      0.00047 &    -653 &              \citet{hoyer_transit_2012} \\
 2454772.96212 &      0.00075 &    -646 &         \citet{fukui_measurements_2011} \\
 2454774.59093 &      0.00030 &    -645 &              \citet{hoyer_transit_2012} \\
 2454787.61792 &      0.00069 &    -637 &              \citet{hoyer_transit_2012} \\
 2455005.82714 &      0.00036 &    -503 &              \citet{hoyer_transit_2012} \\
 2455049.79540 &      0.00080 &    -476 &              \citet{hoyer_transit_2012} \\
 2455075.84947 &      0.00056 &    -460 &             \citet{dragomir_terms_2011} \\
 2455079.10830 &      0.00079 &    -458 &         \citet{fukui_measurements_2011} \\
 2455110.04607 &      0.00089 &    -439 &         \citet{fukui_measurements_2011} \\
 2455123.07611 &      0.00079 &    -431 &         \citet{fukui_measurements_2011} \\
 2455129.58759 &      0.00043 &    -427 &              \citet{hoyer_transit_2012} \\
 2455364.08150 &      0.00110 &    -283 &         \citet{fukui_measurements_2011} \\
 2455377.10955 &      0.00093 &    -275 &         \citet{fukui_measurements_2011} \\
 2455448.75927 &      0.00110 &    -231 &             \citet{dragomir_terms_2011} \\
 2456150.61479 &      0.00056 &     200 &          \citet{moyano_multi-band_2017} \\
 2456150.61396 &      0.00057 &     200 &          \citet{moyano_multi-band_2017} \\
 2458355.50829 &      0.00083 &    1554 &                               This work \\
 2458357.13741 &      0.00071 &    1555 &                               This work \\
 2458358.76412 &      0.00068 &    1556 &                               This work \\
 2458360.39377 &      0.00070 &    1557 &                               This work \\
 2458362.02273 &      0.00073 &    1558 &                               This work \\
 2458363.64908 &      0.00090 &    1559 &                               This work \\
 2458365.27827 &      0.00071 &    1560 &                               This work \\
 2458366.90627 &      0.00075 &    1561 &                               This work \\
 2458370.16411 &      0.00076 &    1563 &                               This work \\
 2458371.79126 &      0.00071 &    1564 &                               This work \\
 2458373.42123 &      0.00075 &    1565 &                               This work \\
 2458375.04910 &      0.00069 &    1566 &                               This work \\
 2458376.67856 &      0.00074 &    1567 &                               This work \\
 2458378.30530 &      0.00087 &    1568 &                               This work \\
 2458379.93419 &      0.00082 &    1569 &                               This work \\
\enddata

%% Include any \tablenotetext{key}{text}, \tablerefs{ref list},
%% or \tablecomments{text} between the \enddata and 
%% \end{deluxetable} commands

%% General table comment marker
\tablecomments{
    $t_{\rm tra}$ is the measured transit midtime, and $\sigma_{t_{\rm tra}}$ is its
    $1\sigma$ uncertainty.
    The ``Reference'' column refers to the work describing the
    original observations.
    All the literature times except for the two \citet{moyano_multi-band_2017}
    times are from the homogeneous \citet{hoyer_transit_2012} analysis.
}

\end{deluxetable}

%% \begin{deluxetable}{} command tell LaTeX how many columns
%% there are and how to align them.
\startlongtable
\begin{deluxetable}{ccccc}
    
%% Keep a portrait orientation

%% Over-ride the default font size
%% Use Default (12pt)
\tabletypesize{\scriptsize}
%% Use \tablewidth{?pt} to over-ride the default table width.
%% If you are unhappy with the default look at the end of the
%% *.log file to see what the default was set at before adjusting
%% this value.

%% This is the title of the table.
\tablecaption{WASP-6b transit times, uncertainties, and references.}
\label{tab:WASP-6b}

%% This command over-rides LaTeX's natural table count
%% and replaces it with this number.  LaTeX will increment 
%% all other tables after this table based on this number
\tablenum{6}

%% The \tablehead gives provides the column headers.  It
%% is currently set up so that the column labels are on the
%% top line and the units surrounded by ()s are in the 
%% bottom line.  You may add more header information by writing
%% another line between these lines. For each column that requries
%% extra information be sure to include a \colhead{text} command
%% and remember to end any extra lines with \\ and include the 
%% correct number of &s.
\tablehead{
  \colhead{$t_{\rm tra}$ [BJD$_\mathrm{TDB}$]} &
  \colhead{$\sigma_{t_{\rm tra}}$ [days]} &
  \colhead{Epoch} & 
  \colhead{Reference}
}

%% All data must appear between the \startdata and \enddata commands
\startdata
 2454425.02167 &      0.00022 &    -398 &        \citet{gillon_discovery_2009} \\
 2455009.83622 &      0.00021 &    -224 &  \citet{tregloan-reed_transits_2015} \\
 2455046.80720 &      0.00015 &    -213 &  \citet{tregloan-reed_transits_2015} \\
 2455073.69529 &      0.00013 &    -205 &  \citet{tregloan-reed_transits_2015} \\
 2455409.79541 &      0.00010 &    -105 &  \citet{tregloan-reed_transits_2015} \\
 2455446.76621 &      0.00058 &     -94 &          \citet{dragomir_terms_2011} \\
 2455473.65439 &      0.00097 &     -86 &     \citet{jordan_ground-based_2013} \\
 2455846.72540 &      0.00045 &      25 &         \citet{sada_extrasolar_2012} \\
 2456088.71801 &      0.00013 &      97 &             \citet{nikolov_hst_2015} \\
 2456095.43974 &      0.00017 &      99 &             \citet{nikolov_hst_2015} \\
 2456132.41082 &      0.00017 &     110 &             \citet{nikolov_hst_2015} \\
 2458357.39410 &      0.00033 &     772 &                            This work \\
 2458360.75573 &      0.00033 &     773 &                            This work \\
 2458364.11691 &      0.00032 &     774 &                            This work \\
 2458370.83872 &      0.00033 &     776 &                            This work \\
 2458374.19952 &      0.00031 &     777 &                            This work \\
 2458377.56026 &      0.00033 &     778 &                            This work \\
 2458380.92185 &      0.00038 &     779 &                            This work \\
\enddata

%% Include any \tablenotetext{key}{text}, \tablerefs{ref list},
%% or \tablecomments{text} between the \enddata and 
%% \end{deluxetable} commands

%% General table comment marker
\tablecomments{
    $t_{\rm tra}$ is the measured transit midtime, and $\sigma_{t_{\rm tra}}$ is its
    $1\sigma$ uncertainty.
    The ``Reference'' column refers to the work describing the
    original observations.
}

\end{deluxetable}

%% \begin{deluxetable}{} command tell LaTeX how many columns
%% there are and how to align them.
\startlongtable
\begin{deluxetable}{ccccc}
    
%% Keep a portrait orientation

%% Over-ride the default font size
%% Use Default (12pt)
\tabletypesize{\scriptsize}
%% Use \tablewidth{?pt} to over-ride the default table width.
%% If you are unhappy with the default look at the end of the
%% *.log file to see what the default was set at before adjusting
%% this value.

%% This is the title of the table.
\tablecaption{WASP-18b transit times, uncertainties, and references.}
\label{tab:WASP-18b}

%% This command over-rides LaTeX's natural table count
%% and replaces it with this number.  LaTeX will increment 
%% all other tables after this table based on this number
\tablenum{7}

%% The \tablehead gives provides the column headers.  It
%% is currently set up so that the column labels are on the
%% top line and the units surrounded by ()s are in the 
%% bottom line.  You may add more header information by writing
%% another line between these lines. For each column that requries
%% extra information be sure to include a \colhead{text} command
%% and remember to end any extra lines with \\ and include the 
%% correct number of &s.
\tablehead{
  \colhead{$t_{\rm tra}$ [BJD$_\mathrm{TDB}$]} &
  \colhead{$\sigma_{t_{\rm tra}}$ [days]} &
  \colhead{Epoch} & 
  \colhead{Reference}
}

%% All data must appear between the \startdata and \enddata commands
\startdata
 2454221.48163 &      0.00038 &   -4037 &    \citet{hellier_orbital_2009} \\
 2455221.30420 &      0.00010 &   -2975 &     \citet{maxted_spitzer_2013} \\
 2455432.18970 &      0.00010 &   -2751 &     \citet{maxted_spitzer_2013} \\
 2455470.78850 &      0.00040 &   -2710 &     \citet{maxted_spitzer_2013} \\
 2455473.61440 &      0.00090 &   -2707 &     \citet{maxted_spitzer_2013} \\
 2455554.57860 &      0.00050 &   -2621 &     \citet{maxted_spitzer_2013} \\
 2455570.58400 &      0.00048 &   -2604 &     \citet{maxted_spitzer_2013} \\
 2455876.55590 &      0.00130 &   -2279 &     \citet{maxted_spitzer_2013} \\
 2456896.14780 &      0.00080 &   -1196 &  \citet{wilkins_searching_2017} \\
 2457255.78320 &      0.00030 &    -814 &  \citet{wilkins_searching_2017} \\
 2457319.80100 &      0.00039 &    -746 &  \citet{wilkins_searching_2017} \\
 2458354.45782 &      0.00016 &     353 &                       This work \\
 2458355.39933 &      0.00015 &     354 &                       This work \\
 2458356.34070 &      0.00018 &     355 &                       This work \\
 2458357.28229 &      0.00018 &     356 &                       This work \\
 2458358.22348 &      0.00018 &     357 &                       This work \\
 2458359.16523 &      0.00020 &     358 &                       This work \\
 2458360.10661 &      0.00017 &     359 &                       This work \\
 2458361.04810 &      0.00017 &     360 &                       This work \\
 2458361.98968 &      0.00016 &     361 &                       This work \\
 2458362.93130 &      0.00018 &     362 &                       This work \\
 2458363.87267 &      0.00018 &     363 &                       This work \\
 2458364.81374 &      0.00017 &     364 &                       This work \\
 2458365.75525 &      0.00019 &     365 &                       This work \\
 2458366.69709 &      0.00018 &     366 &                       This work \\
 2458369.52128 &      0.00017 &     369 &                       This work \\
 2458370.46281 &      0.00017 &     370 &                       This work \\
 2458371.40407 &      0.00017 &     371 &                       This work \\
 2458372.34537 &      0.00018 &     372 &                       This work \\
 2458373.28728 &      0.00018 &     373 &                       This work \\
 2458374.22818 &      0.00016 &     374 &                       This work \\
 2458375.16977 &      0.00017 &     375 &                       This work \\
 2458376.11132 &      0.00018 &     376 &                       This work \\
 2458377.05267 &      0.00017 &     377 &                       This work \\
 2458377.99444 &      0.00018 &     378 &                       This work \\
 2458378.93573 &      0.00016 &     379 &                       This work \\
 2458379.87722 &      0.00017 &     380 &                       This work \\
 2458380.81889 &      0.00018 &     381 &                       This work \\
 2458386.46729 &      0.00016 &     387 &                       This work \\
 2458387.40888 &      0.00017 &     388 &                       This work \\
 2458388.35021 &      0.00016 &     389 &                       This work \\
 2458389.29161 &      0.00015 &     390 &                       This work \\
 2458390.23334 &      0.00016 &     391 &                       This work \\
 2458391.17452 &      0.00016 &     392 &                       This work \\
 2458392.11593 &      0.00016 &     393 &                       This work \\
 2458393.05748 &      0.00015 &     394 &                       This work \\
 2458393.99898 &      0.00016 &     395 &                       This work \\
 2458394.94024 &      0.00017 &     396 &                       This work \\
 2458396.82309 &      0.00015 &     398 &                       This work \\
 2458397.76450 &      0.00015 &     399 &                       This work \\
 2458398.70656 &      0.00016 &     400 &                       This work \\
 2458399.64748 &      0.00015 &     401 &                       This work \\
 2458399.64748 &      0.00015 &     401 &                       This work \\
 2458400.58898 &      0.00017 &     402 &                       This work \\
 2458401.53083 &      0.00016 &     403 &                       This work \\
 2458402.47209 &      0.00017 &     404 &                       This work \\
 2458403.41360 &      0.00016 &     405 &                       This work \\
 2458404.35492 &      0.00017 &     406 &                       This work \\
\enddata

%% Include any \tablenotetext{key}{text}, \tablerefs{ref list},
%% or \tablecomments{text} between the \enddata and 
%% \end{deluxetable} commands

%% General table comment marker
\tablecomments{
    $t_{\rm tra}$ is the measured transit midtime, and $\sigma_{t_{\rm tra}}$ is its
    $1\sigma$ uncertainty.
    The ``Reference'' column refers to the work describing the
    original observations.
    All the literature times are from the homogeneous
    \citet{wilkins_searching_2017} analysis.
}

\end{deluxetable}

%% \begin{deluxetable}{} command tell LaTeX how many columns
%% there are and how to align them.
\startlongtable
\begin{deluxetable}{ccccc}
    
%% Keep a portrait orientation

%% Over-ride the default font size
%% Use Default (12pt)
\tabletypesize{\scriptsize}
%% Use \tablewidth{?pt} to over-ride the default table width.
%% If you are unhappy with the default look at the end of the
%% *.log file to see what the default was set at before adjusting
%% this value.

%% This is the title of the table.
\tablecaption{WASP-46b transit times, uncertainties, and references.}
\label{tab:WASP-46b}

%% This command over-rides LaTeX's natural table count
%% and replaces it with this number.  LaTeX will increment 
%% all other tables after this table based on this number
\tablenum{7}

%% The \tablehead gives provides the column headers.  It
%% is currently set up so that the column labels are on the
%% top line and the units surrounded by ()s are in the 
%% bottom line.  You may add more header information by writing
%% another line between these lines. For each column that requries
%% extra information be sure to include a \colhead{text} command
%% and remember to end any extra lines with \\ and include the 
%% correct number of &s.
\tablehead{
  \colhead{$t_{\rm tra}$ [BJD$_\mathrm{TDB}$]} &
  \colhead{$\sigma_{t_{\rm tra}}$ [days]} &
  \colhead{Epoch} & 
  \colhead{Reference}
}

%% All data must appear between the \startdata and \enddata commands
\startdata
 2455396.60785 &      0.00062 &    -673 &  \citet{anderson_wasp-44b_2012} \\
 2455449.53082 &      0.00026 &    -636 &  \citet{anderson_wasp-44b_2012} \\
 2455722.73178 &      0.00023 &    -445 &    \citet{ciceri_physical_2016} \\
 2455757.06195 &      0.00094 &    -421 &    \citet{petrucci_search_2018} \\
 2455858.61833 &      0.00009 &    -350 &    \citet{ciceri_physical_2016} \\
 2456108.92771 &      0.00094 &    -175 &    \citet{petrucci_search_2018} \\
 2456111.79422 &      0.00016 &    -173 &    \citet{ciceri_physical_2016} \\
 2456111.79413 &      0.00012 &    -173 &    \citet{ciceri_physical_2016} \\
 2456111.79424 &      0.00015 &    -173 &    \citet{ciceri_physical_2016} \\
 2456130.38895 &      0.00042 &    -160 &    \citet{petrucci_search_2018} \\
 2456131.81456 &      0.00112 &    -159 &    \citet{petrucci_search_2018} \\
 2456194.75916 &      0.00027 &    -115 &    \citet{ciceri_physical_2016} \\
 2456217.64127 &      0.00015 &     -99 &    \citet{ciceri_physical_2016} \\
 2456217.64156 &      0.00013 &     -99 &    \citet{ciceri_physical_2016} \\
 2456227.65574 &      0.00060 &     -92 &    \citet{petrucci_search_2018} \\
 2456407.88096 &      0.00015 &      34 &    \citet{ciceri_physical_2016} \\
 2456407.88085 &      0.00018 &      34 &    \citet{ciceri_physical_2016} \\
 2456407.88148 &      0.00028 &      34 &    \citet{ciceri_physical_2016} \\
 2456407.88159 &      0.00043 &      34 &    \citet{ciceri_physical_2016} \\
 2456460.80526 &      0.00017 &      71 &    \citet{ciceri_physical_2016} \\
 2456460.80450 &      0.00024 &      71 &    \citet{ciceri_physical_2016} \\
 2456460.80547 &      0.00064 &      71 &    \citet{ciceri_physical_2016} \\
 2456510.86818 &      0.00060 &     106 &    \citet{petrucci_search_2018} \\
 2456510.86699 &      0.00015 &     106 &    \citet{petrucci_search_2018} \\
 2456516.58667 &      0.00119 &     110 &    \citet{petrucci_search_2018} \\
 2456520.88012 &      0.00064 &     113 &    \citet{petrucci_search_2018} \\
 2456533.75260 &      0.00071 &     122 &    \citet{ciceri_physical_2016} \\
 2456533.75480 &      0.00015 &     122 &    \citet{ciceri_physical_2016} \\
 2456576.66289 &      0.00109 &     152 &    \citet{petrucci_search_2018} \\
 2456589.54197 &      0.00090 &     161 &    \citet{petrucci_search_2018} \\
 2456609.56653 &      0.00043 &     175 &    \citet{petrucci_search_2018} \\
 2456839.85440 &      0.00123 &     336 &    \citet{petrucci_search_2018} \\
 2456862.74085 &      0.00048 &     352 &    \citet{petrucci_search_2018} \\
 2456882.76566 &      0.00073 &     366 &    \citet{petrucci_search_2018} \\
 2456885.62429 &      0.00053 &     368 &    \citet{petrucci_search_2018} \\
 2456915.66040 &      0.00123 &     389 &    \citet{petrucci_search_2018} \\
 2456942.83880 &      0.00078 &     408 &    \citet{petrucci_search_2018} \\
 2456948.56384 &      0.00074 &     412 &    \citet{petrucci_search_2018} \\
 2457274.68458 &      0.00184 &     640 &    \citet{petrucci_search_2018} \\
 2457294.70886 &      0.00140 &     654 &    \citet{petrucci_search_2018} \\
 2457550.74797 &      0.00031 &     833 &    \citet{petrucci_search_2018} \\
 2457593.65692 &      0.00024 &     863 &    \citet{petrucci_search_2018} \\
 2457600.80985 &      0.00039 &     868 &    \citet{petrucci_search_2018} \\
 2457610.82286 &      0.00020 &     875 &    \citet{petrucci_search_2018} \\
 2458326.00972 &      0.00091 &    1375 &                       This work \\
 2458327.43899 &      0.00093 &    1376 &                       This work \\
 2458328.86970 &      0.00094 &    1377 &                       This work \\
 2458330.29965 &      0.00105 &    1378 &                       This work \\
 2458331.73234 &      0.00105 &    1379 &                       This work \\
 2458333.15977 &      0.00086 &    1380 &                       This work \\
 2458334.59230 &      0.00095 &    1381 &                       This work \\
 2458336.02222 &      0.00082 &    1382 &                       This work \\
 2458337.45111 &      0.00099 &    1383 &                       This work \\
 2458340.31143 &      0.00093 &    1385 &                       This work \\
 2458341.74347 &      0.00093 &    1386 &                       This work \\
 2458343.17362 &      0.00093 &    1387 &                       This work \\
 2458344.60303 &      0.00110 &    1388 &                       This work \\
 2458346.03436 &      0.00091 &    1389 &                       This work \\
 2458347.46335 &      0.00168 &    1390 &                       This work \\
 2458348.89621 &      0.00086 &    1391 &                       This work \\
 2458350.32672 &      0.00101 &    1392 &                       This work \\
 2458351.75486 &      0.00103 &    1393 &                       This work \\
\enddata

%% Include any \tablenotetext{key}{text}, \tablerefs{ref list},
%% or \tablecomments{text} between the \enddata and 
%% \end{deluxetable} commands

%% General table comment marker
\tablecomments{
    $t_{\rm tra}$ is the measured transit midtime, and $\sigma_{t_{\rm tra}}$ is its
    $1\sigma$ uncertainty.
    The ``Reference'' column refers to the work describing the
    original observations.
    All the literature times are from the homogeneous
    \citet{petrucci_search_2018} analysis. 14 of the lightcurves
    were acquired by ETD observers \citep[see][]{petrucci_search_2018}.
}

\end{deluxetable}





% \section{Verifying the WASP-4 Archival Times}
% \label{sec:verify_archival_times}
% 
% \begin{figure*}[ht!]
%   \begin{center}
%     \leavevmode
%     \includegraphics[width=0.9\textwidth]{f7.pdf}
%   \end{center}
%   \caption{
%     {\bf Extra times for WASP-4b neither confirm nor refute the
%     	period-decay.}
%     Our timing analysis (Section~\ref{sec:timing}) imposes stringent
%     selection criteria for lightcurves.
%     Here we overplot additional points from the Exoplanet Transit
%     Database (ETD), and the discovery epoch from \citet{wilson_wasp-4b_2008}.
%     \label{fig:verify}
%   }
% \end{figure*}
% 
% The preceding analysis shows that the TESS timestamps are consistent
% at the requisite level with the ${\rm BJD}_{\rm TDB}$ reference.
% Another explanation for the WASP-4b timing variation could be
% incorrect archival times.
% As a sanity check on the times we included for our analysis in
% Table~1, we collected times from the Exoplanet Transit Database
% \citep[ETD; ][]{poddany_ETD_2010}.  We considered only the 12
% lightcurves of the highest self-reported quality (${\rm DQ} = 1$).  We
% then inspected each lightcurve by eye.  If the lightcurve showed
% substantial red noise, we discarded it.  This left 9 midtimes,
% reported in ${\rm HJD}_{\rm UTC}$. We converted the times to ${\rm
% BJD}_{\rm TDB}$ \citep{eastman_achieving_2010}.
% 
% The resulting times are shown in Figure \ref{fig:verify}.  Focusing on
% the epochs that overlap with \citet{huitson_gemini_2017}'s study, four
% of six possible times agree with the \citet{huitson_gemini_2017}
% times; the other two fall about 90 seconds early.  With their stated
% uncertainties, these 6 ETD data points are inconsistent with a
% constant period.  The 6 points came from 5 different observers.
% Systematic errors in absolute clock offsets by different observers
% should be uncorrelated; if there were such systematic errors in the
% ETD times, this could explain the many-sigma outliers.  Though the
% majority of the reported times do support the accuracy of the
% \citet{huitson_gemini_2017} times, a systematic procedure would call
% for averaging the ETD times.  If we did this, and discarded the four
% \citet{huitson_gemini_2017} epochs, it is likely that the evidence for
% period decay would become much weaker.  Since
% \citet{huitson_gemini_2017} documented their time system clearly, and
% produced lightcurves of extremely high quality, we included their times
% in our analysis.  From private correspondence with
% \citealt{huitson_gemini_2017}, the authors also stand by their
% timestamps\todo[inline]{verify this}. Since the ETD data come from
% heterogeneous sources, their timestamps are less clearly documented,
% and the times are thus more prone to systematic errors, we
% categorically omit them from consideration.
% 
% We also overplotted the discovery epoch reported by
% \citet{wilson_wasp-4b_2008}. 
% This epoch folded together WASP-S survey data, a partial FTS
% lightcurve, and a complete EulerCam lightcurve. 
% Though not explicitly stated by \citet{wilson_wasp-4b_2008}, the
% archival WASP-S lightcurves are in ${\rm HJD}_{\rm UTC}$, so they can
% be converted to ${\rm BJD}_{\rm TDB}$ without uncertainty in the
% absolute time reference (Collier-Cameron, priv.\ comm.).
% The discovery epoch falls about 2 minutes earlier than the epochs we
% used for the fit, and is visible in the lower left of Figure
% \ref{fig:verify}.

% Another independent line of evidence supporting the
% \citet{huitson_gemini_2017} times is that \citet{patra_2017} showed
% for WASP-12b that precise transit times reported by
% \citet{stevenson_transmission_2014} from the Gemini telescopes with
% GMOS are congruent with those from other observers.  This indicates
% that the telescopes do not have a long-standing clock offset.


\end{document}
