%\renewcommand{\arraystretch}{1.0}

\startlongtable
\begin{deluxetable*}{lc}

\tabletypesize{\footnotesize}

\tablenum{3}

%\tablewidth{0pt}

\tablecaption{Best-fit transit timing model parameters.}
\label{tab:bestfit}

\tablehead{
  \colhead{Parameter} &
  \colhead{Median Value~(Unc.)\tablenotemark{a}}
}
% provenance:
% /Users/luke/Dropbox/proj/tessorbitaldecay/results/model_comparison/WASP-4b_20200127/model_comparison_output.txt
% processed through inflate_uncertainties.py
\startdata
~~~~~~{\it Constant period} &  \\
$t_0$\,[${\rm BJD}_{\rm TBD}$]    & 2456180.558712(+24)(-24)              \\
$P$\,[days]                       & 1.338231429(+26)(-26)                 \\
~~~~~~{\it Constant period derivative} &  \\
$t_0$~[${\rm BJD}_{\rm TBD}$]     & 2456180.558872(+31)(-31)              \\
$P$\,[days]                       & 1.338231502(+24)(-24)                 \\
$dP/dt$                           & $-2.74(+40)(-40) \times 10^{-10}$     \\
\enddata
\tablenotetext{a}{
The numbers in parenthesis give the $68\%$ confidence interval for the
final two digits, where appropriate.  The intervals have been inflated
by a factor of $(\chi^2_{\rm red})^{1/2}$ due to excess scatter in the
transit residuals (see Section~\ref{sec:transit_analysis}).
}
\vspace{-2cm}
\end{deluxetable*}
