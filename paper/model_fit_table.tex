%\renewcommand{\arraystretch}{1.0}

\startlongtable
\begin{deluxetable}{lc}

\tabletypesize{\footnotesize}

\tablenum{4}

%\tablewidth{0pt}

\tablecaption{Best-fit transit timing model parameters.}
\label{tab:bestfit}

\tablehead{
  \colhead{Parameter} &
  \colhead{Median Value~(Unc.)\tablenotemark{a}}
}

\startdata
~~~~~~{\it Constant period} &  \\
$t_0$\,[${\rm BJD}_{\rm TBD}$]    & 2455804.515752(+19)(-19)              \\
$P$\,[days]                       & 1.338231466(+23)(-22)                 \\
~~~~~~{\it Constant period derivative} &  \\
$t_0$~[${\rm BJD}_{\rm TBD}$]     & 2455804.515918(+24)(-24)              \\
$P$\,[days]                       & 1.338231679(+31)(-31)                 \\
$dP/dt$                           & $-4.00(+37)(-38) \times 10^{-10}$     \\
~~~~~~{\it Apsidal precession (wide prior)} &  \\
$t_0$~[${\rm BJD}_{\rm TBD}$]     & 2455804.51530(+25)(-31)               \\
$P_{\rm s}$\,[days]               & 1.33823127(+20)(-48)                  \\
$e$                               & $1.92^{+1.93}_{-0.76} \times 10^{-3}$ \\
$\omega_0$\,[rad]                 & 2.40(+38)(-34)                        \\
$d\omega/dE$~[rad\,epoch$^{-1}$]  & $8.70^{+3.01}_{-2.30} \times 10^{-4}$ \\
\enddata
\tablenotetext{a}{
The numbers in parenthesis give the $68\%$ confidence interval for the final
two digits, where appropriate.
}
\end{deluxetable}

